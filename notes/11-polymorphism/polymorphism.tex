What if we wanted to be able to sometimes send our mailing list emails using the server's built-in mail program and sometimes using MailChimp?\footnote{We're very deliberately focussing on more abstract classes. A lot of books on OOP focus on classes representing concrete ``things'' in the real world, where there is a strong mapping between what the code does and what the things can physically do. However, it is easy to end up thinking this is the \textit{only} way classes should be used, which inevitably means folks ignore all the OOP principles the second there's a more abstract bit of functionality.} We can't have a type declaration for two different classes, but if we don't limit the type at all then we loose encapsulation.
\\

This is where the idea of \textbf{polymorphism}\footnote{Oh, a heart so true our courage will pull us through you teach me, and I'll teach you Pokemon! Gotta catch 'em all!} comes in. Polymorphism is when two \textit{different} types of object share enough in common that they can take each other's place in a specific context.
\\

There are two ways to enforce this in most OO languages:

\begin{itemize}
    \item \textbf{Interfaces}: when an object implements a defined set of methods.
    \item \textbf{Inheritance}: when an object can inherit methods/properties from another object, creating a hierarchy of object types.
\end{itemize}


\section{Interfaces}

One way we can take advantage of polymorphism is to use ``interfaces''. An \textbf{interface} is a list of \textbf{method signatures} that a class can say it conforms to. It is a contract: if a class implements an interface then we are guaranteed that it has a certain set of methods taking a specified set of arguments.

\pagebreak

Let's have a look at how \texttt{MailingList} uses the object that's passed in:

\begin{minted}{php}
      $mail->to($subscriber)->mail();
\end{minted}

So it needs to be an object that has a \texttt{to()} method that takes a string and returns something that has a \texttt{mail()} method that takes no arguments and doesn't return a value.
\\

We could create an interface for this:

\php{MailInterface.php}{11-polymorphism/figures/01-MailInterface}

The interface consists of \textbf{method signatures}: definitions of the methods that anything implementing the interface must stick to. We should add as much typing information to these as possible: if we didn't include the return type on \texttt{to()} then we couldn't guarantee that it returns something with a \texttt{mail()} method.
\\

We can then say that \texttt{Mail} \textbf{implements} this interface:

\php{Mail.php}{11-polymorphism/figures/02-Mail}

Our class already has matching methods, so this should work without needing to change anything. If we created a class that doesn't fully implement an interface then your code won't even run: you'll get an error straight away.


\pagebreak


Finally, we need to update \texttt{MailingList}:

\begin{minted}{php}
    public function send(MailerInterface $mail)
    {
        //...as before
    }
\end{minted}

Rather than using the \texttt{Mail} type declaration we've used the \textit{interface} type instead. Our code should work as before.
\\

But now we could also pass in \textit{any} class that implements \texttt{MailInterface}. Which means we could create a \texttt{MailChimp} class:

\php{MailChimp.php}{11-polymorphism/figures/03-MailChimp}

And use it instead:

\php{}{11-polymorphism/figures/04-app}

The \texttt{Mail} and \texttt{MailChimp} classes are \textbf{polymorphic} in respect to how they can be used inside \texttt{MailingList}. Either class in guaranteed to work because they both implement \texttt{MailerInterface}, which guarantees that the code in \texttt{MailingList} that uses the passed-in object will work.
\\

It's worth noting that we \textit{only} declare method signatures that are used inside \texttt{MailingList}: we could add method signatures for the \texttt{from()}, \texttt{subject()}, and \texttt{message()} methods too, but that would just unnecessarily limit the sorts of things that we can pass in. After all, we only use the \texttt{to()} and \texttt{mail()} methods, so those are the only ones we need to guarantee encapsulation.
\\

A class can \texttt{implement} as many interfaces as it likes:

\begin{minted}{php}
    class MailChimp implements
        MailInterface,
        HttpInterface,
        InterspaceInterface
    {
        //... code here
    }
\end{minted}


\subsection{Message Passing}

Interfaces are a core idea in OOP. But one that is often not talked about with beginner level books on OOP. Although it's called \textit{Object}-Oriented Programming, it's not actually the objects that should get the focus, it's the \textbf{messages} that they can send to one another: the methods and their parameters.
\\

The reason interfaces are such a powerful idea is because they focus solely on the messages and don't tell you anything about the implementation. This is important because of encapsulation. If we have to worry about how a specific object does something, then we can't treat it as a black box.
\\

This is why we only declare \texttt{public} \textit{methods} in an interface: \texttt{private} methods and properties are about how the class is implemented, whereas interfaces are all about how the classes talk to one another.
\\


\quoteinline{I'm sorry that I long ago coined the term ``objects'' for this topic because it gets many people to focus on the lesser idea. The big idea is ``messaging'' \textellipsis{} The key in making great and growable systems is much more to design how its modules communicate rather than what their internal properties and behaviours should be}{Alan Kay, Father of OOP}



\section{Inheritance}

Inheritance lets us create a hierarchy of object types, where the \textbf{children} types inherit all of the methods and properties of the \textbf{parent} classes. This allows reuse of methods and properties but allowing for different behaviours.
\\

For example, both \texttt{Mail} and \texttt{MailChimp} will be responsible for sending an email, so they will likely have some things behaviours in common (the \texttt{to()}, \texttt{from()}, \texttt{message()}, and \texttt{subject()} methods for example).
\\

We could create a parent \texttt{Mailer} class that contains all of this repeated functionality:

\php{Mailer.php}{11-polymorphism/figures/05-Mailer}

Notice that the properties are all \texttt{protected}: if they were \texttt{private} they would not be accessible inside any of the children classes.
\\

We don't put the \texttt{mail()} method into the \texttt{Mailer} class, as it will be different for each implementation. We can then \textbf{extend} the \texttt{Mailer} class to copy its behaviour into \texttt{Mail} and \texttt{MailChimp}:

\php{Mail.php}{11-polymorphism/figures/06-Mail}

\pagebreak

Now, in the \texttt{MailingList} class we can use \texttt{Mailer} as the type declaration. That means that depending on our mood,\footnote{Or possibly something more concrete} we can send messages with either the local mail server or with MailChimp:

\begin{minted}{php}
    public function send(Mailer $mail)
    {
        //...as before
    }
\end{minted}

A child class inherits all of the types of its parents, so a child of \texttt{Mailer} will pass this type check.
\\

Again, we have polymorphism: different classes with some shared functionality that means they can be used interchangeably in a specific context.


\subsection{Abstract Classes}

But you can perhaps see a few issues with this. Firstly, you could create an instance of \texttt{Mailer} and pass that into \texttt{send()}. This would cause an issue because the \texttt{Mailer} class doesn't have a \texttt{mail()} method, so you'd get an error. Secondly, there's no guarantee that a child of \texttt{Mailer} has a \texttt{mail()} method: we could easily create a child of \texttt{Mailer} but forget to add it. This would lead to the same issue:

\begin{minted}{php}
    $mailinglist = new MailingList();

    // oops, won't work
    // Mailer doesn't have a send method
    // so we'll get an error when send()
    // tries to call it
    $mailinglist->send(new Mailer());
\end{minted}


This is where \textbf{abstract classes} come in. These are classes that are not meant to be used to create object instances directly, but instead are designed to be the parent of other classes. We can setup properties and methods in them, but they can't actually be constructed, only extended.
\\

We can also create \texttt{abstract} method signatures. These allow us to say that a child class \textit{has} to implement a method with the given name and parameters. We'll get an error if we don't.
\\

This gets round both issues: we won't be able to instance \texttt{Mailer} and we can make sure any of its children have a \texttt{mail()} method:

\php{Mailer.php}{11-polymorphism/figures/07-MailerRedux}


\subsection{Overriding}

Children classes can \textbf{override} methods and properties from parent classes. That means if we wanted to make it so that the \texttt{to()} method in the \texttt{MailChimp} class did something slightly different, then we could write a different \texttt{to()} method. As long as it has \textit{the same method signature}\footnote{The \texttt{\_\_construct()} method is an exception to this rule. As it is unique to the specific class it can have a different set of parameters.} (i.e. excepts the same parameters), this will work:

\pagebreak

\php{MailChimp.php}{11-polymorphism/figures/08-override}

If necessary it's possible to stop a child class from overriding a method by adding the \texttt{final} keyword in front of it:

\php{Mailer.php}{11-polymorphism/figures/09-final}


\subsection{\texttt{parent}}

Sometimes when we're overriding a method it can be useful to still have access to the parent object's version. For example, we might want to override the \texttt{to()} method of \texttt{Mailer}, but only to add a bit of functionality. We can do this by calling the method name on \texttt{parent}. Make sure you pass along the arguments when you do this:

\php{MailChimp.php}{11-polymorphism/figures/10-parent}

You can call the parent's constructor method with \texttt{parent::\_\_construct()}.



\subsection{Inheritance Tax}

\quoteinline{The object-oriented version of ``Spaghetti code'' is, of course, ``Lasagna code'': too many layers}{Roberto Waltman}

If you read any books about OOP they'll focus a lot of their time on inheritance. While inheritance can be very useful, all of this attention means that it's often the technique that programmers reach for when they need to add similar functionality to multiple classes. And it will almost always lead to much more complicated code.
\\

If you think about it, inheritance can easily break encapsulation. With poorly written inheritance structures it can sometimes be necessary to understand the inner-workings of all a class's parent classes. It can lead to situations where you become scared to change a line of code in the parent class in case it inadvertently affects a piece of code in one of the children classes. And the last thing you want to be writing is code that you're worried about changing.\footnote{This wisdom courtesy of many such experiences.}
\\

That's not to say inheritance isn't useful. It's totally fine to inherit well-documented code that frameworks or libraries provide, as in these cases you're generally adding one tiny bit of functionality to something that's much more complicated under-the-hood.\footnote{Although some purists would say even in this case there are better alternatives: see the \href{https://www.thoughtfulcode.com/orm-active-record-vs-data-mapper/}{Active Record vs Data Mapper} debate} But if it's code that you've written, it's always worthwhile thinking ``Do I really need to use inheritance?''
\\

Sandi Metz\footnote{She's been doing OOP since it was invented in the 70s, so she probably knows what she's talking about} suggests not using inheritance until you have \textit{at least} three classes that are \textit{definitely} all using exactly the same methods. You should never start out by writing an abstract class: write the actual use-cases first and only write an abstract class if you definitely need one. If you do use inheritance then try not to create layers and layers of it: maybe have a rule that you'll only ever inherit through one layer.

\section{Interfaces plus Inheritance}

Any class can implement an interface, including parent classes. So our \texttt{Mailer} class could implement the \texttt{MailInterface}:

\begin{minted}{php}
    abstract class Mailer implements MailInterface
    {
        // ...as before
    }
\end{minted}

Then the \texttt{MailingList} class could still use \texttt{MailInterface} as the type declaration:

\begin{minted}{php}
    public function send(MailerInterface $mail)
    {
        //...as before
    }
\end{minted}

In general, interfaces are more flexible than inheritance, so they should be preferred for type declarations.


\section{Composition}

\quoteinline{Prefer composition over inheritance}{The Gang of Four, \textit{Design Patterns}}

The idea of \textit{composition over inheritance} is that rather than sharing behaviour with inheritance, we share it using interfaces and shared classes. If a lot of classes share the same implementation of a method, rather than using inheritance consider moving it into a separate class that they can all share. It might require a bit more code to get working, but it is much easier to make changes to.
\\

For example, in the case of \texttt{Mail} and \texttt{MailChimp} we could pull the email specific code out in an \texttt{Email} class, which both classes can use. This separates the \textit{generating} of an email from the \textit{sending} of an email, which is probably \href{https://en.wikipedia.org/wiki/Single-responsibility_principle}{no bad thing}.



\section{Additional Resources}

\begin{itemize}[leftmargin=*]
    \item \href{https://phpapprentice.com/interfaces.html}{PHP Apprentice: Interfaces}
    \item \href{https://phpapprentice.com/classes-inheritance.html}{PHP Apprentice: Inheritance}
    \item \href{https://en.wikipedia.org/wiki/Composition\_over\_inheritance}{Wikipedia: Composition over Inheritance}
    \item \href{https://www.poodr.com}{Practical Object-Oriented Design in Ruby}: An intermediate book about OOP. In Ruby, but all the key concepts work in PHP too.
    \item \href{https://en.wikipedia.org/wiki/SOLID}{Wikipedia: The SOLID Principles of OOP}: Quite advanced, but incredibly useful if you can get your head round it.
    \item \href{https://www.youtube.com/channel/UCk3yOoaVtORwXipuLZ3jWNg}{YouTube: Sandi Metz Videos}: If you're doing OOP in a few years time, watch all of these. There's a lifetime of experience in these talks.
\end{itemize}
