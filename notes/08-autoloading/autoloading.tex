We only know how to use PHP when everything is in one \texttt{.php} file. As we start to build more complex code with multiple classes it would be good to keep things separate. Ideally we'd have one \texttt{class} per file in an easy to find location.


\section{Autoloading}

Currently, if you put one of your classes in a separate PHP file and then try and use it from another PHP file you'll get an error:

\begin{minted}{php}
    <?php

    $app = new App(); // PHP Fatal error: Class 'App' not found
\end{minted}

We could require the file this is in manually:

\begin{minted}{php}
    <?php

    require_once __DIR__ . "/classes/App.php";

    $app = new App(); // success!
\end{minted}

But this will only work for a few files before it becomes tedious.
\\

Thankfully, modern PHP supports \textbf{autoloading}: we can programmatically tell PHP how to find a class that it doesn't know about yet. Technically we could do this in whatever way we like:

\php{}{08-autoloading/figures/01-autoload}

But it's generally better if everyone does things in the same way, which is where the \href{https://www.php-fig.org/psr/psr-4/}{PSR-4 Autoloader standard} comes in.
\\

But before we can understand that, we need to know about \textbf{namespaces}.


\section{Namespaces}

Namespaces solve the problem of needing to work with two or more classes that happen to have the same name. For example, in a large app you might have a \texttt{Post} class that represents a blog post, but you might also have included a Slack package that includes a \texttt{Post} class that posts messages to Slack. It would be cumbersome to make sure that every single class used in a large app had a unique class name.
\\

The most everyday use of namespaces is the file system on your computer: you can have two files called exactly the same thing \textit{as long as they're in separate directories}.
\\

Namespacing in PHP is much the same idea. We assign each class to a namespace and then we can have two classes with the same name, \textit{as long as they're in separate namespaces}. This means that when we use the class we need to tell PHP which namespace we are talking about.
\\

We assign a namespace by adding a \texttt{namespace} declaration as the very first line of PHP:

\begin{minted}{php}
    <?php

    namespace Blog\Data;

    class Post { ... }
\end{minted}

Now, when we want to use this class we'll need to use the namespace:

\begin{minted}{php}
    new Blog\Data\Post();
\end{minted}

\pagebreak

If we use a class many times inside another file, we can put a \texttt{use} statement at the top to tell it to always use a specific namespaced class:

\begin{minted}{php}
    use Blog\Data\Post;

    // further down the file
    new Post(); // actually new Blog\Data\Post()

    // we can use the other namespaced Post class
    // we just need to use the full namespace
    new Services\Slack\Post();
\end{minted}

Generally we'll use the same class multiple times inside a file, so this saves a lot of typing.
\\

If the class you want to use is in the \textit{same} namespace as the current class you don't even need a \texttt{use} statement.
\\

You can \textbf{alias} a class to give it a different name in the file you're working in. This can be particularly useful if you have two classes which share the same class name but are in different namespaces:

\begin{minted}{php}
    use Blog\Data\Post;
    use Services\Slack\Post as SlackPost;

    new Post(); // actually new Blog\Data\Post()
    new SlackPost(); // actually new Services\Slack\Post()
\end{minted}


\pagebreak

\begin{infobox}{The static \texttt{class} property}
    Sometimes it's useful to get the fully namespaced name of a class as a string (e.g. in a configuration file). All classes have a static \texttt{class} property:

    \begin{minted}{php}
        $class = Services\Slack\Post::class;
        dump($class); // "Services\Slack\Post"
    \end{minted}

    This is particularly useful as backslashes need escaping in strings, meaning if you were to write the string out by hand you'd have to write:

    \begin{minted}{php}
        $class = "Services\\Slack\\Post"; // need to escape every backslash
    \end{minted}

    It also means you'll get an error if you try and use a class that doesn't exist.
\end{infobox}




\section{PSR-4 Autoloading}

The PSR-4 autoloading spec basically ties namespaces to a directory structure within a specified directory. Composer can do all the work of autoloading for us, but we do need to set it up.
\\

First, create a directory called \texttt{app}. Then edit the \texttt{composer.json} so that it looks like this:\footnote{Your \texttt{require} section might look somewhat different}

\code{composer.json}{json}{08-autoloading/figures/02-composer.json}

Here we've told Composer that any namespace starting with the root \texttt{App} should look for files in the \texttt{app} directory. We could use anything as the root namespace or directory name, but this is the convention.
\\

Next, run \texttt{composer dump}: this regenerates the Composer autoload file for us\footnote{Technically, it gets rid of the existing autoload file and then creates a new one – hence the name}.
\\

Now, as long as we stick to the following rules, we won't need to require any other files manually:

\begin{enumerate}
    \item One class per file, where the file name is the same as the class name (case sensitive)
    \item Put all of our classes in the \texttt{App} root namespace
    \item If we add directories inside the \texttt{app} directory (for extra organisation), they add an extra level to the namespace
\end{enumerate}

Because namespaces and classes in PHP are usually capitalised and, with PSR-4, the directory and filenames match the namespace/class naming, all the files and directories inside \texttt{app} will also be capitalised.
\\

For example, if we had a class called \texttt{Post} that just sat inside the \texttt{app} directory, it should be in a file called \texttt{Post.php} and have the namespace \texttt{App\textbackslash Post}. If we had a class \texttt{Post} that did something with Slack we could create a directory \texttt{app/Slack} and then put the file \texttt{Post.php} in it with the namespace \texttt{App\textbackslash Slack\textbackslash Post}.


\section{Additional Resources}

\begin{itemize}[leftmargin=*]
    \item \href{http://www.php.net/manual/en/language.namespaces.rationale.php}{PHP: Namespaces Overview}
    \item href{https://www.php-fig.org/psr/psr-4/}{PSR-4: Autoloader Spec}
\end{itemize}
