Composer is PHP's \textbf{package manager} (like \texttt{npm} is for JavaScript). It lets us easily add code written by other people to our projects.
\\

One of Composer's responsibilities is to set up autoloading of any libraries that it adds, that way you don't have to manually link to all the files that you use in your project. It's also very easy to setup Composer so it will autoload any classes that \textit{you've} created.


\section{Initialising}

First we need to add Composer. Run the following in the project directory that you want to add Composer to:

\begin{minted}{bash}
    # the -n bit stops it asking you a bunch of questions
    composer init -n
\end{minted}

This will add a \texttt{composer.json} file to your project.



\section{PSR-4 Autoloading}

Next we need to tell Composer to load our classes for us. We're going to use the \href{https://www.php-fig.org/psr/psr-4/}{PSR-4 namespace standard}. Basically, this means that we pick a directory to be the ``root'' of our namespace, and everything from that point on is just based on directory names.
\\

First, create a directory called \texttt{app}. Then edit the \texttt{composer.json} so that it looks like this:

\begin{minted}{json}
{
    "autoload": {
       "psr-4": {"App\\": "app/"}
    },
    "require": {}
}
\end{minted}

Here we've told Composer that any namespace starting with the root \texttt{App} should look for files in the \texttt{app} directory. We could use anything as the root namespace or directory name.
\\

Next, run \texttt{composer dump}: this, somewhat confusingly, generates an autoload file for us\footnote{Technically, it gets rid of the existing autoload file and then creates a new one - hence the name} in a directory called \texttt{vendor} (this is where Composer installs any libraries we might want to use).
\\

We'll also need to create a file in the root of our project that looks like this:

\begin{minted}{php}
require_once __DIR__ . '/vendor/autoload.php';

// ... code that uses the classes
\end{minted}

Now, as long as we stick to the following rules, we won't need to require any other files manually:

\begin{enumerate}
    \item One class per file, where the file name is the same as the class name (case sensitive)
    \item Put all of our classes in the \texttt{App} root namespace
    \item If we add directories inside the \texttt{app} directory (for extra organisation), they add an extra level to the namespace
\end{enumerate}

Because namespaces and classes in PHP are usually capitalised and, with PSR-4, the directory and filenames match the namespace/class naming, all the files and directories inside \texttt{app} will also be capitalised.
\\

For example, if we had a class called \texttt{Post} that just sat inside the \texttt{app} directory, it should be in a file called \texttt{Post.php} and have the namespace \texttt{App\textbackslash Post}. If we had a class \texttt{Post} that did something with Slack we could create a directory \texttt{app/Slack} and then put the file \texttt{Post.php} in it with the namespace \texttt{App\textbackslash Slack\textbackslash Post}.



\section{Libraries}

As well as handling autoloading for us, Composer's main purpose is to let us use bits of code other people have written. Let's take a look at some useful libraries.
\\

You can search for libraries on \href{https://packagist.org}{Packagist} or \href{https://libraries.io/search?languages=PHP}{Libraries.io}.


\begin{infobox}{Composer \& Git}
    You don't want to add the \texttt{vendor} directory to your Git repositories, as it can be easily recreated by running \texttt{composer install}. So make sure you add \texttt{vendor/} to your \texttt{.gitignore} file.
\end{infobox}


\subsection{\texttt{symfony/var-dumper}}

We install this with \texttt{composer require symfony/var-dumper}. We then have access to the \texttt{dump()} function, which is a more useful version of \texttt{echo}:

\begin{minted}[startinline=true]{php}
dump([1, 2, 3, 4]);
/*
    array:4 [
      0 => 1
      1 => 2
      2 => 3
      3 => 4
    ]
*/

dump(new Person("Zazu"));
/*
    App\Person {
      -name: "Zazu"
    }
*/
\end{minted}

It also adds the \texttt{dd()} function (for ``debug and die''), which dumps the result and then immediately stops the PHP. This can be useful if you want to check something half-way through a process. Be careful, if you use \texttt{dd()} nothing after it will run.
\\

See the \href{https://symfony.com/doc/current/components/var_dumper.html}{VarDumper documentation} for more information.


\subsection{\texttt{Illuminate\textbackslash Support\textbackslash Collection}}

This is part of Laravel, which we'll be covering later. It basically lets us handle arrays in a way which isn't utterly horrible.
\\

We install it by running \texttt{composer require illuminate/support}. It's got \href{http://laravel.com/docs/master/collections#available-methods}{tonnes of really useful methods}, but we'll just look at four of them here: our old friends \texttt{map()}, \texttt{filter()}, and \texttt{reduce()}, as well as a very useful one called \texttt{pluck()}.
\\

Generally collection methods return a new collection object. You can turn a \texttt{Collection} back into a standard array by calling its \texttt{all()} method.

\subsubsection{\texttt{filter}}

Filter is almost identical to JavaScript: we pass it an anonymous function that takes each item in the array and returns a boolean value. It returns a new \texttt{Collection} containing all the items for which the function returned \texttt{true}:

\phpinputminted{03/figures/02/01-filter}

\subsubsection{\texttt{map}}

Map is also very similar to JavaScript: we pass it an anonymous function that takes each item in the array and transforms the value somehow. It returns a new \texttt{Collection} where each item has been transformed:

\phpinputminted{03/figures/02/02-map}

\subsubsection{\texttt{reduce}}

Again, reduce is very similar to JavaScript: we pass it an anonymous function that takes the accumulated value and each item in the array. The return value is passed in as the accumulator value for the next iteration. It returns the final accumulated value:

\phpinputminted{03/figures/02/03-reduce}

Make sure you pass in an initial value for the accumulator, otherwise it will be \texttt{null}, which might cause problems.

\subsubsection{\texttt{pluck}}

We've not come across \texttt{pluck} before, but it's very useful. It assumes your collection contains either associative arrays or objects all with the same structure. You pass it a key value and it extracts a new \texttt{Collection} contain just that key/property from each item in the collection:

\phpinputminted{03/figures/02/04-pluck}



\subsection{\texttt{nesbot/carbon}}

The Carbon library makes working with dates in PHP much easier. We install it by running \texttt{composer require nesbot/carbon}\footnote{If you've already installed \texttt{illuminate/support} this isn't necessary, as it's a dependency for that package}. Once it's installed we have access to the \texttt{Carbon\textbackslash Carbon} class.

\phpinputminted{03/figures/02/05-Carbon}

There are many more features listed in the \href{https://carbon.nesbot.com/docs/}{Carbon documentation}.


\subsection{\texttt{fzaninotto/Faker}}

\href{https://github.com/fzaninotto/Faker}{Faker} is a library that generates fake data for you. This can be useful for ``seeding'' databases: generating test data to make sure that everything's working. We install it with \texttt{composer require fzaninotto/faker}.

\phpinputminted{03/figures/02/06-Faker}

Faker supports \href{https://github.com/fzaninotto/Faker#formatters}{hundreds of different types of data}: IP addresses, credit card numbers, dates, colours, images, \&c.


\begin{infobox}{The Case of the Missing Students?}
    When building the student preparation app for the course we created lots of fake users to make sure everything was working. Then, when real students started signing up, they didn't appear in the admin interface. It took a bit of sleuthing to work out what was going wrong: the query to fetch the students should have been checking for the student ``type'' field, but in actual fact was checking the student's name for the word ``student''. And it just so happened that every single test student that we'd added to the database had the name ``Test Student <number>''! If we'd been using Faker to generate random names we'd have noticed the issue earlier!
\end{infobox}

\section{Additional Resources}

\begin{itemize}[leftmargin=*]
    \item \href{https://www.php-fig.org/psr/psr-4/meta/}{More about PSR-4}
    \item \href{https://getcomposer.org}{Composer}
\end{itemize}
