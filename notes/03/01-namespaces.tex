So far we've had to put all of our classes into a single file, when ideally we'd like \textit{one class per file}.\footnote{It makes your code much more reusable}

\section{\texttt{require\_once}}

We can import one PHP file into another using the \texttt{require\_once} keyword:

\phpinputminted{03/figures/01/01-require-once}

We give \texttt{require\_once} a relative file path and it will be as if the contents of that file are included in place.
\\

\begin{infobox}{\texttt{require} and \texttt{include}}
    There is also the \texttt{require} command: this does the same as \texttt{require\_once}, except you could accidentally load the same file in more than once - which you almost never want to do.
    \\

    There is also \texttt{include} and \texttt{include\_once} which do the same as the \texttt{require} equivalents, except if it can't find the file it will keep running and just show an error message. However, normally if we're trying to include a file we wouldn't want the code to run at all if it can't find the file, so \texttt{require} is preferred.
\end{infobox}



\section{Naming Collisions}

Large PHP apps can have hundreds (or even thousands) of classes. It's not uncommon for two classes to end up with the same name. For example, in a blog app you might have a \texttt{Post} class which deals with the data for each post on the site. But you might also have a \texttt{Post} class which posts an update to Slack each time a post is added.
\\

We could, of course, be careful about the naming of each class, calling one \texttt{BlogPost} and the other \texttt{SlackPost}, but in large apps it can be tricky keeping track of every class name that you've used - and it becomes practically impossible when you have multiple developers working on the same app.
\\

Even if we're really careful naming our classes, we don't have any control over the names of classes in PHP libraries that other people have written. It would be unrealistic to make sure that none of the class names you've used clash with those in any libraries that you might use.


\begin{infobox}{The Bad Old Days}
    I lied just then. Back in the before-times, when PHP was still trying to find itself, you \textit{did} have to use unique names for every single class - including the ones in libraries (which you had no control over). To get around this issue you would pick an almost definitely unique prefix (like your company name) and add it to the front of every single class in the app: \texttt{SmallHadronCollider\_BlogPost}, \texttt{SmallHadronCollider\_SlackPost}. Needless to say, this made the code where the classes were used very messy.
\end{infobox}



\section{Namespaces}

\textbf{Namespaces} were added to PHP 5.3 to avoid this problem. The most everyday use of namespaces is the file system on your computer: you can have two files called exactly the same thing \textit{as long as they're in separate directories}.
\\

Namespacing in PHP is much the same idea. We assign each class to a namespace and then we can have two classes with the same name, \textit{as long as they're in separate namespaces}. This means that when we use the class we need to tell PHP which namespace we are talking about.
\\

We assign a namespace by adding a \texttt{namespace} declaration at the top of the file:

\begin{minted}{php}
    namespace Blog\Data;

    class Post { ... }
\end{minted}

Now, when we want to use this class we'll need to use the namespace:

\begin{minted}{php}
    new Blog\Data\Post();
\end{minted}

This might not seem any better than the old way of doing things (i.e. using \texttt{BlogPost}), but PHP also gives us the \texttt{use} keyword.
\\

We can put a \texttt{use} statement at the top of a PHP file to tell it to always use a specific namespaced class:

\begin{minted}{php}
    use Blog\Data\Post;

    // further down the file
    new Post(); // actually new Blog\Data\Post()

    // we can use the other namespaced Post class
    // we just need to use the full namespace
    new Services\Slack\Post();
\end{minted}

Generally we'll use the same class multiple times inside a file, so this saves a lot of typing.
\\

If the class you want to use is in the \textit{same} namespace as the current class you don't even need a \texttt{use} statement.
\\

You can \textbf{alias} a class to give it a different name in the file you're working in. This can be particularly useful if you have two classes which share the same class name but are in different namespaces:

\begin{minted}{php}
    use Blog\Data\Post;
    use Services\Slack\Post as SlackPost;

    new Post(); // actually new Blog\Data\Post()
    new SlackPost(); // actually new Services\Slack\Post()
\end{minted}


\begin{infobox}{The static \texttt{class} property}
    Sometimes it's useful to get the fully namespaced name of a class as a string (e.g. in a configuration file). All classes have a static \texttt{class} property:

    \begin{minted}{php}
        $class = Services\Slack\Post::class;
        var_dump($class); // string(19) "Services\Slack\Post"
    \end{minted}

    This is particularly useful as backslashes need escaping in strings, meaning if you were to write the string out by hand you'd have to write:

    \begin{minted}{php}
        $class = "Services\\Slack\\Post"; // need to escape every backslash
    \end{minted}

    It also means you'll get an error if you try and use a class that doesn't exist.
\end{infobox}





\section{Autoloading}

When I said earlier that PHP apps can contain thousands of classes you might have thought ``Well that's going to be an awful lot of \texttt{require\_once} statements''. And, in fact, historically that's exactly what you'd have: a file called something like \texttt{load.php} which listed thousands of files. Every time you wanted to add a class you'd need to write it and then make sure you added it to the massive list.
\\

Thankfully, things have moved on since then and PHP supports \textbf{autoloading}. This lets us tell PHP where to find a specific class based on its name and namespace. However, writing this code ourselves is unnecessary because we'd be much better using the Composer package manager to do it for us.

\section{Additional Resources}

\begin{itemize}[leftmargin=*]
    \item \href{https://daylerees.com/php-namespaces-explained/}{PHP Namespaces Explained}
    \item \href{http://php.net/manual/en/language.oop5.autoload.php}{Auto-Loading Classes}
\end{itemize}
