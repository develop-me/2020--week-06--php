\section{Dates}

PHP makes working with dates quite pleasant as long as you use the \texttt{DateTime} class (and its relatives).

\begin{minted}{php}
    // a date now
    $dt = new DateTime();

    // output as a string
    $dt->format("Y-m-d H:i:s"); // e.g. "2020-02-04 15:13:59"

    // a date for a specific time
    $dt = new DateTime("5th January 2020, 12:04:03");
    $dt->format("d/m/Y @ H.i.s"); // "05/01/2020 @ 12.04.03"
\end{minted}

The \texttt{format} method takes a string. You can find out all the possible parts on \href{http://www.php.net/manual/en/function.date.php}{date documentation}.
\\

You can also do date maths using the \texttt{DateInterval} class:

\begin{minted}{php}
    // an interval representing:
    // 2 years, 3 months, 5 days, 6 hours, 2 minutes and 6 seconds
    $interval = new DateInterval("P2Y3M5DT6H2M6S");

    // a date now
    $dt = new DateTime();

    // add an interval to an existing date
    // new DateTime, 2 years, 3 months (etc.) in the future
    $future = $dt->add($interval); // new DateTime object
    $future->format("Y-m-d H:i:s"); // e.g. "2022-05-09 21:24:08"

    // subtract interval from a date
    $past = $dt->sub($interval); // new DateTime object

    // work out the difference between two dates
    $birthday = new DateTime("1984-04-16");
    $difference = $dt->diff($birthday);

    // difference in years
    $difference->y; // 35
\end{minted}

The period format is pretty strange for \texttt{DateInterval}. It starts with a \texttt{P} (for ``period''), then using \texttt{Y}, \texttt{M}, and \texttt{D} for year, month, and day. Then there's a \texttt{T} for ``time'' followed by \texttt{H}, \texttt{M}, and \texttt{S} for hour, minute, and second.

\hr

Although PHP's date functionality is better than some, it's still fairly common to use a library like \texttt{Carbon} to deal with dates and times in PHP.
\\

Whatever you do, avoid the \texttt{date()}, \texttt{mktime()}, and \texttt{strtotime()} functions like the plague!

\section{Additional Resources}

\begin{itemize}[leftmargin=*]
    \item \href{http://www.php.net/manual/en/class.datetime.php}{PHP: The \texttt{DateTime} Class}
    \item \href{http://www.php.net/manual/en/class.dateinterval.php}{PHP: The \texttt{DateInterval} Class}
\end{itemize}
