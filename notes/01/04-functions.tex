Functions in PHP serve the same purpose as functions in JavaScript: they allow us to reuse bits of code.
\\

They are written in the same way as functions were historically written in most C-style language (including JavaScript):

\phpinputminted{01/figures/04/01-function}


\section{Scope}

Because we don't declare variables in PHP it takes a very explicit approach to scope: by default variables used inside functions are assumed to be locally scoped.
\\

The following will cause an error as \texttt{\$message} is assumed to be in local scope:

\phpinputminted{01/figures/04/02-scope-no-global}

This means if we want to access a non-local variable we need to use the \texttt{global} keyword:

\phpinputminted{01/figures/04/03-scope-global}

This can actually be seen as a positive, as it makes it harder to write impure functions by accident.


\section{Anonymous Functions}

Standard functions in PHP are not values like they are in JavaScript, so we can't just pass them round and assign them to variables in quite the same way. However, passing functions around is such a useful thing to be able to do that recent versions of PHP added \textbf{closures} as a way to do this:

\begin{minted}{php}
    $add = function ($a, $b) {
        return $a + $b;
    };

    $result = $add(1, 2);
    var_dump($result); // int(3)
\end{minted}

As you can see, because it's stored in a variable we need to use the \texttt{\$} when calling the function.
\\

Some functions in PHP require a \texttt{callable} argument, which just means a function:

\begin{minted}{php}
    $result = array_map(function ($value) {
        return $value * $value;
    }, [1, 2, 3, 4, 5]);

    var_dump($result); // [1, 4, 9, 16, 25]
\end{minted}

Closures don't have access to variables declared outside of themselves. In order to use these you use the \texttt{use} keyword:

\begin{minted}{php}
    $message = "Hello";

    $say = function ($a) use ($message) {
        return "{$message} {$a}";
    };

    $result = $say("Wombat");
    var_dump($result); // string(12) "Hello Wombat"
\end{minted}


\section{Strict Types}

As a general rule functions should only accept arguments of a specific type and return arguments of a specific type: e.g. \texttt{add} should only accept numbers. We can enforce this using ``type declarations''\footnote{In previous versions of PHP these were called ``type hints''}.
\\

First we turn on strict typing:

\begin{minted}{php}
    <?php

    declare(strict_types=1); // always first line after opening tag
\end{minted}

Next we add type declarations to our function:

\begin{minted}{php}
    function add(float $a, float $b) : float {
        return $a + $b;
    }
\end{minted}

Each parameter (\texttt{\$a} and \texttt{\$b}) is given an explicit type. We also add a ``return'' type - the type of value the function returns - after the parameter brackets.
\\

The parameter types and return types needn't match. Here's a function that takes a string (\texttt{\$str}) and integer (\texttt{\$times}) as parameters and returns a string:

\begin{minted}{php}
    function repeat(string $str, int $times) : string {
        $output = "";

        for ($i = 0; $i < $times; $i += 1) {
            $output .= $string;
        }

        return $output;
    }
\end{minted}

The possible type declarations are:

\begin{itemize}
    \item \texttt{int}: for numbers that have to be whole (e.g. a limit of a \texttt{for} loop)
    \item \texttt{float}: for any numbers (an integer passed to \texttt{float} \textit{will} work)
    \item \texttt{string}
    \item \texttt{bool}
    \item \texttt{array}
    \item \texttt{callable}: a closure
\end{itemize}

If you try and call a function and pass in/return the wrong type of value PHP will throw an error.


\subsection{\texttt{void}}

If a function doesn't return anything, we can use the special \texttt{void} return type:

\begin{minted}{php}
    // function doesn't return anything
    // so use void return type
    function echoNewLine(string $str) : void {
        echo "{$str}\n";
    }
\end{minted}

Generally it's best to write functions that do return useful values, but there are some cases where this is not practical.


\section{Additional Resources}

\begin{itemize}[leftmargin=*]
    \item \href{http://www.php.net/manual/en/functions.user-defined.php}{PHP: Functions}
    \item \href{http://www.php.net/manual/en/language.types.callable.php}{PHP: Callable}
    \item \href{http://www.php.net/manual/en/functions.arguments.php#functions.arguments.type-declaration.strict}{PHP: Strict Typing}
\end{itemize}
