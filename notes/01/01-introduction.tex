PHP is a general purpose programming language that is frequently used to run the ``server-side'' code for websites. It was originally created as a simple ``templating language'', based loosely on Perl, to allow outputting HTML with repeated elements. Over the years it has evolved into a fully object-oriented programming language.
\\

PHP should look quite familiar if you've done JavaScript as they are both ``C-based'' languages, meaning that they share a syntax style: semi-colons, curly braces, and brackets.
\\

For now we're just going to run PHP in the command-line like we did in Week 3 with JavaScript. PHP doesn't have a REPL built-in like Node, so we always need to run a file:

\begin{minted}{bash}
    php file.php # run the given file
\end{minted}


\section{Hello, World!}

As is tradition, we should write a ``Hello, World!'' program before moving forward:

\inputminted{php}{01/figures/01/01-hello.php}

We ``echo'' the string ``Hello, World!'', this is PHP's equivalent of using \texttt{console.log()} - although it's only useful when used with strings. \texttt{echo} is \textit{not} a function (or method), but instead a \textbf{keyword}. All that really means is that you don't \textit{call} it.
\\

As you can see, it's not that different from JavaScript. The main difference is that on line 1 we have to tell PHP that it's PHP. If we don't do this then PHP has a bit of an identity crisis and doesn't know what's going on.
\\

\textbf{Make sure you never have anything before the opening \texttt{<?php} tag - no spaces, line breaks, or comments - as this will break most modern PHP apps.}


\begin{infobox}{Ye Olde PHP}
    The opening \texttt{<?php} tag can seem a bit strange in modern PHP, as it doesn't seem to serve any purpose - surely it knows it's PHP?
    \\

    When PHP was used mainly as a templating language PHP files would be made up mostly of HTML with only snippets of PHP:

    \begin{minted}{html}
        <body>
            <h1><?php echo $header ?></h1>
            <div>
                <?php echo $content ?>
            </div>
        </body>
    \end{minted}

    PHP is still used like this in some PHP frameworks such as WordPress. Nowadays templating languages like Blade and Twig are more commonly used, thus separating the program logic from the templating language.
\end{infobox}

In JavaScript you could get away without semi-colons at the end of lines. PHP isn't nearly so forgiving: if you forget a semi-colon you will get a syntax error and the code will refuse to run.
\\

\textbf{From now on code examples won't include the opening tag, but you will need to add it as the first line in all your files.}


\section{\texttt{var\_dump}}

As mentioned above \texttt{echo} isn't that useful for values that aren't strings, so we'll be using \texttt{var\_dump} instead. This outputs not only the value we're interested but also relevant information about it.
\\

For example, running the following:

\begin{minted}{php}
    var_dump("Hello");
    var_dump(12 + 12);
    var_dump(true);
\end{minted}

Would give us:

\begin{minted}{text}
    string(5) "Hello"
    int(24)
    bool(true)
\end{minted}

\texttt{var\_dump} \textit{is} a function, so we call it and pass it values.
\\

\texttt{var\_dump} isn't brilliant for working with complex values, so we'll switch to using a nicer way of doing it once we've learned up using PHP libraries.


\section{Variables}

In PHP we don't \textit{declare} variables, we just start using them. This is possible because variables have to start with a \texttt{\$} (this is because of PHP's Perl influence).
\\

This makes it much easier to accidentally change the values of existing variables, as there is no difference between creating a new variable and reassigning an existing one. So be very careful when naming things.

\phpinputminted{01/figures/01/02-variables}

Variables must start with a dollar, followed by a letter or underscore, followed by any number of letters, numbers, or underscores. Variables names are case-sensitive.
\\

As with JavaScript it is a standard convention to use \texttt{\$camelCase} for variable names in PHP, although you may see \texttt{\$snake\_case} used in older code.


\begin{infobox}{Documentation: A Warning}
    Be careful using the official PHP documentation: \textit{anyone} can submit ``User Contributed Notes'' and they often contain code samples that are to be avoided. This is often because the notes were added years ago when PHP was a very different language.
    \\

    As a general rule, don't look at the ``User Contributed Notes'' section: use Stack Overflow if the documentation hasn't cleared it up for you.
\end{infobox}



\section{Additional Resources}

\begin{itemize}[leftmargin=*]
    \item \href{https://www.php.net/manual/en/function.var-dump.php}{PHP: \texttt{var\_dump}}
    \item \href{http://php.net/manual/en/language.variables.basics.php}{PHP: Variable Basics}
    \item \href{https://en.wikipedia.org/wiki/PHP}{Wikipedia: PHP}
    \item \href{https://en.wikipedia.org/wiki/Perl}{Wikipedia: Perl}
\end{itemize}
