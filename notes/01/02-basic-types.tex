PHP has the same basic types as JavaScript: numbers, strings, and booleans. They work in much the same way.


\section{Numbers}

Unlike JavaScript PHP does have a sense of whether a number is an integer (whole number) or a ``floating point number'' (one with a decimal place). This means that some of the issues we had with JavaScript and decimal numbers don't cause issues in PHP:

\phpinputminted{01/figures/02/01-numbers}


The operators should be familiar:
\\

\begin{small}
    \begin{tabu}{c l X}
        \textbf{Operator} & \textbf{Name} & \textbf{Description} \\
        \texttt{+}  & addition        & adds two numbers together \\
        \texttt{-}  & subtraction     & subtracts the second number from the first number \\
        \texttt{*}  & multiplication  & multiplies two numbers \\
        \texttt{/}  & division        & divides the first number by the second \\
        \texttt{\%} & modulus         & remainder after dividing the first number by the second
    \end{tabu}
\end{small}

\par\bigskip


We don't have a \texttt{Math} object in PHP, instead there are just lots of functions - but the naming should be familiar:

\phpinputminted{01/figures/02/02-number-functions}


But we can still run into weird issues with numbers in PHP. As a general rule \textit{never trust floating point numbers}

\begin{minted}{php}
    floor((0.1 + 0.7) * 10); // 7 - oop!
\end{minted}


\section{Strings}

Strings are also very similar to JavaScript.
\\

You can use single- or double-quotes:

\begin{minted}{php}
    $firstName = "Casper";
    $lastName = 'Spoooky';
\end{minted}

Double-quotes allow you to interpolate values:

\begin{minted}{php}
    $fullName = "{$firstName} {$lastName}";
\end{minted}

We use curly-braces to enter interpolation mode and then use the variable name inside. The \texttt{\$} here is part of the variable, whereas in JavaScript interpolation it's part of the syntax for interpolation (so the dollar is \textit{outside} the curly-braces).
\\

Because interpolation is such a common thing to do, generally we'll use double-quotes for strings.

\subsection{Concatenation}

PHP uses \texttt{.} for concatenation, this avoids the issues that JavaScript had with the overloaded \texttt{+} operator. It can, however, lead to many a brain-fart as you try and use \texttt{+} when you mean \texttt{.}:

\begin{minted}{php}
    var_dump($firstName . $lastName); // "CasperSpooky"
    var_dump($firstName + $lastName); // PHP Warning - A non-numeric value encountered
    var_dump("1" + "2"); // 3 - coerces to numbers
\end{minted}

\subsection{String Functions}

Unlike in JS, strings are not objects in PHP. That means they don't have properties or methods. So we have to use functions to work with them:

\begin{minted}{php}
    strtolower("Blah"); // "blah"
    strtoupper("Blah"); // "BLAH"
    trim("   Blah  "); // "Blah"
    substr("Fishsticks", 4); // "sticks"
\end{minted}

There are many other string function in the \href{http://www.php.net/manual/en/ref.strings.php}{PHP documentation}.


\section{Booleans}

As with JavaScript, PHP has the boolean values \texttt{true} and \texttt{false}. The only difference in PHP is that they're not case sensitive:

\begin{minted}{php}
    $bool = true;
    $bool = True; // also valid
    $bool = TRUE; // still valid
    $bool = TrUe; // totes valid
\end{minted}

For consistency it's best to stick with the lowercase version.

\subsection{Boolean Logic}

PHP has \texttt{\&\&}, \texttt{||}, and \texttt{!} which work in the same way as in JavaScript:

\begin{minted}{php}
    true && false; // false
    true || false; // true
    !true; // false
    !false; // true
\end{minted}

PHP also has the written versions \texttt{and} and \texttt{or}:

\begin{minted}{php}
    true and false; // false
    true or false; // true
\end{minted}

If you're coming from JavaScript, you should use the \texttt{\&\&} and \texttt{||} versions. The other versions have different ``precedence'' and will not always work as you expect.

\subsection{Comparison Operators}

All your favourite comparison operators are back:
\\

\begin{small}
    \begin{tabu}{c l X}
        \textbf{Operator} & \textbf{Name} & \textbf{Description} \\
        \texttt{===} & strict equality & \texttt{true} if the values are the same \\
        \texttt{!==} & non-equality & \texttt{false} if the values are the same\\
        \texttt{<} & less than & \texttt{true} if the first value is less than the second value  \\
        \texttt{>} & greater than & \texttt{true} if the first value is greater than the second value\\
        \texttt{<=} & less than or equal to & \texttt{true} if the first value is less than or equal to the second value  \\
        \texttt{>=} & greater than or equal to & \texttt{true} if the first value is greater than or equal to the second value
    \end{tabu}
\end{small}

\par\bigskip

As with JavaScript there are also the type-coercing \texttt{==} and \texttt{!=} operators, but these are best avoided.


\subsection{Falsy Values}

PHP has all the falsy values that JavaScript has. Empty arrays are also falsy in PHP (which they are \textit{not} in JavaScript):

\begin{itemize}
    \item \texttt{false} itself
    \item The number zero: \texttt{0}, \texttt{0.0}, \texttt{-0}, \texttt{-0.0}
    \item The empty string: \texttt{""}
    \item An empty array: \texttt{[]}
    \item \texttt{NULL}
\end{itemize}



\section{Additional Resources}

\begin{itemize}[leftmargin=*]
    \item \href{https://www.php.net/manual/en/language.operators.arithmetic.php}{PHP: Arithmetic Operators}
    \item \href{https://www.php.net/manual/en/ref.math.php}{PHP: Maths Functions}
    \item \href{http://php.net/manual/en/language.operators.logical.php}{PHP: Logical Operators}
    \item \href{http://php.net/manual/en/language.operators.comparison.php}{PHP: Comparison Operators}
\end{itemize}
