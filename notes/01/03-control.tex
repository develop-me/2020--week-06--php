Because they're both C-based languages, loops and conditionals in PHP and JavaScript are syntactically identical. Just remember that variables in PHP are not declared and always start with a \texttt{\$}.

\section{Conditionals}

\subsection{\texttt{if} Statements}

A basic \texttt{if} statement:

\begin{minted}{php}
    if ($x < 10) {
        // do a thing
    }
\end{minted}

With an \texttt{else}:

\begin{minted}{php}
    if ($x < 10) {
        // do a thing
    } else {
        // do the other thing
    }
\end{minted}

An \texttt{else if}:

\begin{minted}{php}
    if ($x < 10) {
        // do a thing
    } else if ($x < 20) {
        // do this thing
    } else {
        // do the other thing
    }
\end{minted}

In PHP the space between \texttt{else} and \texttt{if} can be omitted:

\begin{minted}{php}
    if ($x < 10) {
        // do a thing
    } elseif ($x < 20) {
        // do this thing
    } else {
        // do the other thing
    }
\end{minted}

Although they're not technically necessary for single line blocks, you should \textit{always} use the curly braces around a block.


\subsection{Ternary Operator}

PHP also has the ternary operator. As with JavaScript, a ternary operator is an \textit{expression}:

\begin{minted}{php}
    // if $index is less than 0, set it to 5
    // otherwise decrement it
    $index = $index < 0 ? 5 : $index - 1;
\end{minted}

\subsection{\texttt{switch} Statements}

\texttt{switch} statements are also available:

\begin{minted}{php}
    switch ($x) {
        case 1:
            $message = "It's One";
            break;
        case 2:
            $message = "It's Two";
            break;
        default:
            $message = "No idea";
    }
\end{minted}

Don't forget to \texttt{break} at the end of each \texttt{case}; and, remember, \texttt{switch} statements are only useful if you want to run different bits of code based on the same expression.


\section{Loops}

\subsection{\texttt{for} Loops}

\texttt{for} loops are much the same as in JavaScript (except we don't declare the counter variable and there's dollars everywhere):

\begin{minted}{php}
    $total = 0;

    for ($i = 1; $i <= 10; $i += 0) {
        $total += $i;
    }

    var_dump($total); // int(55)
\end{minted}

Remember three parts:

\begin{enumerate}
    \item Setup the counter variable (runs once before the loop starts)
    \item Run the loop as long as this is \texttt{true}
    \item Evaluated after each iteration
\end{enumerate}

\subsection{\texttt{while} Loops}

While loops are useful if you don't know how many times the loop needs to run:

\begin{minted}{php}
    $i = 0;
    $total = 0;

    while ($total < 100) {
        $i += 1;
        $total += $i;
    }

    var_dump($total); // int(105)
\end{minted}

\subsection{\texttt{do-while} Loops}

A \texttt{do-while} loop is the same as a \texttt{while} loop, except that the \texttt{do} block will always run at least once. They can sometimes be a little easier to work out.

\begin{minted}{php}
    $i = 0;
    $total = 0;

    do {
        $i += 1;
        $total += $i;
    } while ($total < 100);

    var_dump($total); // int(105)
\end{minted}

\hr

As with all loops, be careful not to create an infinite loop!


\section{Additional Resources}

\begin{itemize}[leftmargin=*]
    \item \href{http://www.php.net/manual/en/control-structures.if.php}{PHP: \texttt{if}}
    \item \href{http://www.php.net/manual/en/control-structures.elseif.php}{PHP: \texttt{else if}}
    \item \href{https://www.php.net/manual/en/language.operators.comparison.php#language.operators.comparison.ternary}{PHP: The Ternary Operator}
    \item \href{http://www.php.net/manual/en/control-structures.switch.php}{PHP: \texttt{switch}}
    \item \href{http://www.php.net/manual/en/control-structures.for.php}{PHP: \texttt{for} Loops}
    \item \href{http://www.php.net/manual/en/control-structures.while.php}{PHP: \texttt{while} loops}
    \item \href{http://www.php.net/manual/en/control-structures.do.while.php}{PHP: \texttt{do-while} loops}
\end{itemize}
