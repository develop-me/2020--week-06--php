Arrays comes in two forms in PHP: numerically indexed and associative.
\\


\section{Numerically Indexed Arrays}

Numerically indexed arrays are exactly the same as in JavaScript.

\begin{minted}{php}
    $values = [1, 2, 3, 4, 5];
    $first = $values[0]; // first item in array
    $last = $values[4]; // last item in this array
    var_dump($first); // 1
    var_dump($last); // 5
\end{minted}

We can add a value to the end of an array:

\begin{minted}{php}
    $values[] = 6;
\end{minted}

We can also find out how many items are in an array:

\begin{minted}{php}
    count($values); // 5
\end{minted}

\begin{infobox}{Old Skool Arrays}
    If you're working with older PHP you may see arrays written using the older notation:

    \begin{minted}{php}
        $values = array(1, 2, 3, 4, 5);
    \end{minted}

    This isn't actually a function, although it does look like one.
    \\

    The newer notation was added in PHP 5.4, which has reached end-of-life, so you should not need to use the older notation.
\end{infobox}

\section{Associative Arrays}

Associative arrays are effectively PHP's version of object literals: in PHP ``objects'' are always instances of a \texttt{class}, but the key-value pairing of object literals is still a useful concept.

\begin{minted}{php}
    $assoc = [
        "firstName" => "Ben",
        "lastName" => "Wales",
        "dob" => "2018-08-24",
    ];

    var_dump($assoc["lastName"]); // string(5) "Wales"
\end{minted}

\begin{infobox}{All Arrays Are Associative}
    Technically speaking, all arrays in PHP are actually associative, they're just automatically given a numerical key if you don't provide one. This does mean you can run into some unusual results if you're not careful:

    \begin{minted}{php}
        $arr = [];
        $arr["key"] = 1; // key provided
        $arr[] = 2; // no key, so starts at 0

        var_dump($arr); // ["key" => 1, 0 => 2]
    \end{minted}

    You should not deliberately mix numerically indexed and associative arrays as things can get weird.
\end{infobox}


\section{Iterating Over Arrays}

We can use a \texttt{foreach} loop to iterate over every item in an array.

\begin{minted}{php}
    $values = [1, 2, 3, 4, 5];

    foreach ($values as $value) {
        // $value will be each value in turn
    }
\end{minted}

You can also get the key:

\begin{minted}{php}
    $assoc = [
        "firstName" => "Ben",
        "lastName" => "Wales",
        "dob" => "2018-08-24",
    ];

    foreach ($assoc as $key => $value) {
        // $key will be each key in turn
        // $value will be each value in turn
    }
\end{minted}

As numerical indexes are just associative arrays with numerical keys, you can also use this syntax to get the index of a numerical array:

\begin{minted}{php}
    $values = [1, 2, 3, 4, 5];

    foreach ($values as $index => $value) {
        // $index will be each index in turn
        // $value will be each value in turn
    }
\end{minted}




\section{Additional Resources}

\begin{itemize}[leftmargin=*]
    \item \href{http://www.php.net/manual/en/language.types.array.php}{PHP: Arrays}
    \item \href{https://www.php.net/manual/en/ref.array.php}{PHP: Array Functions}
    \item \href{http://www.php.net/manual/en/control-structures.foreach.php}{PHP: \texttt{foreach}}
\end{itemize}
