Arrays comes in two forms in PHP: numerically indexed and associative.
\\


\section{Numerically Indexed Arrays}

Numerically indexed arrays are exactly the same as in JavaScript.

\begin{minted}{php}
    $values = [1, 2, 3, 4, 5];
    $first = $values[0]; // first item in array
    $last = $values[4]; // last item in this array
    var_dump($first); // 1
    var_dump($last); // 5
\end{minted}

We can add a value to the end of an array:

\begin{minted}{php}
    $values[] = 6;
\end{minted}

We can also find out how many items are in an array:

\begin{minted}{php}
    count($values); // 5
\end{minted}


\section{Associative Arrays}

Associative arrays are effectively PHP's version of object literals: in PHP ``objects'' are always instances of a \texttt{class}, but the key-value pairing of object literals is still a useful concept.


\section{Iterating Over Arrays}

We can use a \texttt{foreach} loop to iterate over every item in an array.



\section{Additional Resources}

\begin{itemize}[leftmargin=*]
    \item \href{https://www.php.net/manual/en/ref.array.php}{PHP: Array Functions}
\end{itemize}
