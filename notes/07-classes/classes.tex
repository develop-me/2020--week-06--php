In PHP (and all object-oriented languages), an \textbf{object} is a way of storing some values (\textbf{properties}) and functionality (\textbf{methods}) that are related in some way.
\\

It's very common to need many objects that have the same methods and property names, but where the \textit{values} of the properties are specific to each object.
\\

For example we might have two objects, that both represent people: both have a \texttt{firstName} property, a \texttt{lastName} property, and a \texttt{birthdate} property as well as a \texttt{getAge()} method, which returns their age based on the date of birth and the current date. The property names and methods will be identical for both objects, however the \textit{values} for the properties will be different: a ``Mark'' object would have \texttt{"Mark"}, \texttt{"Wales"}, and \texttt{"1984-04-16"} for the respective values, whereas a ``Ben'' object would have \texttt{"Ben"}, \texttt{"Wales"}, and \texttt{"2018-08-24"}.
\\

\img{25em}{07-classes/img/objects}{1.2em}{Two different objects with the same properties, but different values}

\pagebreak

A \textbf{class} is an abstract representation of an object that you want to create. For example, you might have a class \texttt{Person} that allows you to create lots of object \textbf{instances} representing different people.
\\

Here's a class that represents a person:

\php{}{07-classes/figures/01-Person}

Here's how we'd use our class:

\php{}{07-classes/figures/02-usage}

You can see that where we'd write a dot in JavaScript (\texttt{ben.getAge()}), we write an arrow in PHP (\texttt{\$ben->getAge()}), but otherwise it's almost identical in usage.

\begin{infobox}{PSR-2: Coding Style Guide}
    You've possibly noticed that in all the examples above the opening curly brace (\texttt{\{}) for classes and methods is on its own line. This is part of the \href{https://www.php-fig.org/psr/psr-2/}{PSR-2: Coding Style Guide} spec.
    \\

    If you do an \texttt{if} statement (or other control structure) then the opening curly brace, obviously, goes on \textit{the same} line.
    \\

    You're probably thinking that this doesn't make the slightest bit of sense. And you'd be right. PSR-2 was created by sending round a questionnaire about coding style to 30 or so of the most prolific PHP programmers and they just went with whatever the majority said for each point.
    \\

    But it's the style that everyone uses now. You'll get used to it.
\end{infobox}

\section{Properties}

In most OOP languages we can declare the properties that our object has\footnote{JavaScript is still relatively immature as a ``classical'' OOP language, but support for this is coming}. We normally declare all of our properties at the top of the class: this makes it easy to see in one glance what properties the object instance will have.

\php{}{07-classes/figures/03-properties}

We declare properties by using the \texttt{private}\footnote{More on this later} keyword followed by the name of the property – with a dollar at the front, just like with variables. However, properties are \texttt{not} variables, as they belong to a specific object instance: each object instance with its own set of values.
\\

You should always get and set properties using methods (``getters'' and ``setters''). We'll go into why this is later in the week.


\subsection{Default Values}

We can assign properties default values when we declare them. For example, we could add an \texttt{\$arms} property to the \texttt{Person} class and set it to \texttt{2} by default:

\php{}{07-classes/figures/04-property-values}

You can always use a method to change the value later.
\\

If you declare a property but don't assign it a value it will have the value \texttt{null}.


\section{\texttt{\$this}}

Inside our classes we can use the \texttt{\$this} keyword to access properties and methods that belong to the current object instance. It works in much the same way as JavaScript except that it's much more reliable: \texttt{\$this} in PHP \textit{always} refers to the current object and has no meaning elsewhere.

\php{}{07-classes/figures/05-Address}

Notice that it's the \texttt{\$this} bit that has the dollar in front of it, the property name does not, even though they \textit{do} have a dollar in front of them when we declare them at the top of the class\footnote{For weird historical reasons}.

\section{Constructor}

When we create a new object instance with \texttt{new} any values that we pass as arguments will be given to the \texttt{\_\_construct()} method, just as if we'd called it ourselves. So the number of arguments that we pass with \texttt{new} should always match the number of parameters that the \texttt{\_\_construct()} method accepts:

\php{}{07-classes/figures/06-construct}

We need to make sure we store the values passed in as properties on the newly created instance, otherwise they'll cease to exist once the \texttt{\_\_construct()} method has completed\footnote{Remember, parameters only exist inside the function they belong to}. It's fairly standard to name the parameters the same thing as the properties that they'll be assigned to as in the above example.
\\

If we've given a property a default value we don't need to set it up in \texttt{\_\_construct()} as well, it will automatically be given the default value when the object instance is created. If \textit{all} of your properties have default values then a \texttt{\_\_construct()} method might not even be necessary\footnote{It's not uncommon in full OOP code to accept no parameters in the constructor and to use setters for everything.}.

\pagebreak

\begin{infobox}{Using \texttt{new} Objects}
    Unlike in JavaScript you can't immediately use a created object:

    \begin{minted}{php}
        // won't work
        new Person("Jim", "Henson", "1936-09-24")->getAge();
    \end{minted}

    If you really want to, you can get around this with a pair of brackets around the \texttt{new} statement:

    \begin{minted}[frame=topline]{php}
        // will work
        // but you don't have a reference to the object anymore
        (new Person("Jim", "Henson", "1936-09-24"))->getAge();
    \end{minted}

    However, this isn't something you're likely to need to do all that often.
\end{infobox}


\section{Additional Resources}

\begin{itemize}[leftmargin=*]
    \item \href{https://phpapprentice.com/classes.html}{PHP Apprentice: Classes}
    \item \href{https://laracasts.com/series/php-for-beginners/episodes/12}{Laracasts: Classes 101}
    \item \href{https://laracasts.com/series/object-oriented-bootcamp-in-php/episodes/1}{Laracasts: Classes}
\end{itemize}
