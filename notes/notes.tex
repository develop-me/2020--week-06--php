% setup the document
\documentclass[b5paper,openany]{book}

% setup variables ------------------------------

% your name
\newcommand\authors{Mark Wales}

% the week number
\newcommand\weekno{6}

% week main title
\newcommand\maintitle{Object-Oriented Programming}

% week subtitle
\newcommand\subtitle{with PHP}

% link to notes - relative to github.com/develop-me/
\newcommand\github{bootcamp--week-06--php/blob/master/notes/}

% setup functions, styling, etc.
\input{../../latex-templates/template.tex}

% the structure of the document
\begin{document}

% render the title page (uses variables above)
\tp

\quotepage{Certainly not every good program is object-oriented, and not every object-oriented program is good.}{Bjarne Stroustrup, Creator of C++}

\tableofcontents

\input{../../latex-templates/preface.tex}

\chapter{PHP}
\quoteinline{I don't know how to stop it, there was never any intent to write a programming language \textellipsis{} I have absolutely no idea how to write a programming language, I just kept adding the next logical step on the way.}{Rasmus Lerdorf, Creator of PHP}

PHP is a general purpose programming language that is frequently used to run the \textbf{server-side} code for websites. It was originally created as a simple \textbf{templating language}, based loosely on Perl, to allow outputting HTML with repeated elements – PHP originally stood for ``Personal Home Page''\footnote{More recently ``PHP: Hypertext Preprocessor''}. But, over the years, it has evolved into a fully \textbf{object-oriented programming language}.
\\

PHP should look quite familiar if you've done JavaScript as they are both ``C-based'' languages, meaning that they share a syntax style: semi-colons, curly braces, and brackets.
\\

We can run PHP in the command-line similarly to how we did in Week 3 with JavaScript:

\begin{minted}{bash}
    php file.php # run the given file
\end{minted}


\section{Hello, World!}

As is tradition, we should write a ``Hello, World!'' program before moving forward:

\begin{minted}{php}
    <?php

    echo "Hello, World!";
\end{minted}

We ``echo'' the string ``Hello, World!'', this is PHP's equivalent of using \texttt{console.log()} – although it's only useful when used with strings. \texttt{echo} is \textit{not} a function (or method), but instead a \textbf{keyword}. All that really means is that you don't \textit{call} it.
\\

As you can see, it's not that different from JavaScript. The main difference is that on line 1 we have to tell PHP that it's PHP. If we don't do this then PHP has a bit of an identity crisis and doesn't know what's going on.
\\

\textbf{Make sure you never have anything before the opening \texttt{<?php} tag – no spaces, line breaks, or comments – as this will break most modern PHP apps.}


\begin{infobox}{Ye Olde PHP}
    The opening \texttt{<?php} tag can seem a bit strange in modern PHP, as it doesn't seem to serve any purpose – surely it knows it's PHP?
    \\

    When PHP was used mainly as a templating language PHP files would be made up mostly of HTML with only snippets of PHP:

    \begin{minted}{html}
        <body>
            <h1><?php echo $header ?></h1>
            <div>
                <?php echo $content ?>
            </div>
        </body>
    \end{minted}

    PHP is still used like this in some PHP frameworks such as WordPress. Nowadays templating languages like Blade and Twig are more commonly used, thus separating the program logic from the templating language.
\end{infobox}

In JavaScript you could get away without semi-colons at the end of lines. PHP isn't nearly so forgiving: if you forget a semi-colon you will get a syntax error and the code will refuse to run.
\\

\textbf{From now on code examples won't include the opening tag, but you will need to add it as the first line in all your files.}


\section{Composer}

Composer is the PHP package manager. It lets us easily download code that other people have written and use it in our own code. Although it's technically not necessary to use it at this point, we're going to be using Composer to make our PHP experience more pleasant.

\subsection{REPL}

First, let's get a PHP REPL, as there isn't one built in:

\begin{minted}{bash}
    composer global require psy/psysh
\end{minted}

This will install \href{https://psysh.org/}{PsySH}. You only need to run this once on your machine. The \texttt{global} bit means that it will install it so that you can use it anywhere on your computer.
\\

You should now be able to run \texttt{psysh} from any directory to get a PHP REPL up and running. This can be useful for quickly checking bits of code.
\\

Just type \texttt{exit} if you want to get back to the command-line.

\subsection[Taking a Dump]{Taking a Dump\footnote{Teehee}}

As mentioned above \texttt{echo} is actually pretty useless for anything other than strings\footnote{Try \texttt{echo}ing \texttt{true} if you don't believe me}. If we want to be able to log out any data type, then we'll need to use something else.
\\

PHP does have the \texttt{var\_dump()} function, which will log out different data types, but it's still fairly horrible.

\begin{minted}{php}
    dump(12); // int(12)
    dump([1, 2, 3]); // array(3) { [0]=> int(1) [1]=> int(2) [2]=> int(3) }
\end{minted}

We're going to use the \texttt{symfony/var-dumper} package\footnote{Part of the \href{https://symfony.com/}{Symfony framework}, but written so that it can be used with any PHP code} to allow us to get behaviour much more similar to \texttt{console.log()} in JS. We'll need to install this on a \textbf{per-project} basis. To do this, first choose the directory that you want to work from, and then run:

\begin{minted}{bash}
    composer require symfony/var-dumper
\end{minted}

This will create \texttt{composer.json} and \texttt{composer.lock} files and a \texttt{vendor} directory.

\begin{infobox}{Composer \& Git}
    The \texttt{composer.json} tracks which packages you've installed (as well as other bits of Composer configuration). The \texttt{composer.lock} file keeps track of the \textit{exact} versions of packages that have been installed so that another developer could recreate an identical set of packages.
    \\

    You can recreate your \texttt{vendor} directory by running \texttt{composer install} after cloning a repository.
    \\

    As such, your \texttt{composer.json} and \texttt{composer.lock} files \textit{should} go into version control. Your \texttt{vendor} directory should \textit{not} go into version control. So you should immediately create a \texttt{.gitignore} file when adding Composer:

    \code{.gitignore}{text}{01-php/figures/01-gitignore}

\end{infobox}

To use the package (and any other packages that we install) we need to let PHP know to use the Composer files. We do this by adding the following line to the top of any PHP files that need to use it:

\begin{minted}{php}
    // load in the Composer configuration
    // __DIR__ just means the directory this file is in
    require __DIR__ . "/vendor/autoload.php";
\end{minted}

We now have access to the glamorously named \texttt{dump()} function. \texttt{dump()} is our \texttt{console.log()} equivalent: it outputs things nicely and automatically adds syntax highlighting to the output. Having installed the package, we could rewrite our ``Hello, World!'' example as follows:

\begin{minted}{php}
    <?php

    require __DIR__ . "/vendor/autoload.php";

    dump("Hello, World!");
\end{minted}

We'll be using \texttt{dump()} from now on.


\section{Variables}

In PHP we don't \textit{declare} variables, we just start using them. This is possible because variables have to start with a \texttt{\$} (this is because of PHP's Perl influence).
\\

This makes it much easier to accidentally change the values of existing variables, as there is no difference between creating a new variable and reassigning an existing one. So be very careful when naming things.

\begin{minted}{php}
    $name = "Archie";

    $age = 4;
    $houseNumber = 21;

    $name = "Ben"; // changes the value of $name - deliberate?

    // using variables
    $notUseful = $age + $houseNumber; // 25
\end{minted}

Variables must start with a dollar, followed by a letter or underscore, followed by any number of letters, numbers, or underscores. Variable names are case-sensitive.
\\

As with JavaScript it is a standard convention to use \texttt{\$camelCase} for variable names in PHP, although you may see \texttt{\$snake\_case} used in older code.


\section{Documentation: A Warning}

Be careful using the official PHP documentation: \textit{anyone} can submit ``User Contributed Notes'' and they often contain code samples that are to be avoided. This is often because the notes were added years ago when PHP was a very different language.
\\

As a general rule, don't look at the ``User Contributed Notes'' section: use Stack Overflow if the documentation hasn't cleared it up for you.



\section{Additional Resources}

\begin{itemize}[leftmargin=*]
    \item \href{https://getcomposer.org/}{Composer}
    \item \href{https://psysh.org/}{PsySH}
    \item \href{https://symfony.com/doc/current/components/var_dumper.html}{Symfony: The VarDumper Component}
    \item \href{http://php.net/manual/en/language.variables.basics.php}{PHP: Variable Basics}
    \item \href{https://en.wikipedia.org/wiki/PHP}{Wikipedia: PHP}
    \item \href{https://en.wikipedia.org/wiki/Perl}{Wikipedia: Perl}
\end{itemize}


\chapter{Basic Types}
PHP has the same basic types as JavaScript: numbers, strings, and booleans (also known as the ``Scalar'' types\footnote{Meaning they represent a single value}). They work in much the same way.


\section{Numbers}

Unlike JavaScript PHP does have a sense of whether a number is an integer (whole number) or a ``floating point number'' (one with a decimal place). This means that some of the issues we had with JavaScript and decimal numbers don't cause issues in PHP:

\begin{minted}{php}
    dump(12 + 12); // 24
    dump(0.1 + 0.2); // 0.3 - hurrah!
\end{minted}


The operators should be familiar:
\\

\begin{small}
    \begin{tabularx}{\textwidth}{c l X}
        \textbf{Operator} & \textbf{Name} & \textbf{Description} \\
        \texttt{+}  & addition        & adds two numbers together \\
        \texttt{-}  & subtraction     & subtracts the second number from the first number \\
        \texttt{*}  & multiplication  & multiplies two numbers \\
        \texttt{/}  & division        & divides the first number by the second \\
        \texttt{\%} & modulus         & remainder after dividing the first number by the second
    \end{tabularx}
\end{small}

\pagebreak

We don't have a \texttt{Math} object in PHP, instead there are just lots of functions - but the naming should be familiar:

\begin{minted}{php}
    // rounding
    floor(12.3030); // 12
    ceil(12.3030); // 13
    round(12.3030); // 12

    // powers/roots
    pow(2, 3); // 8
    sqrt(16); // 4

    // random
    mt_rand(5, 10); // random integer between 5 and 10 (inclusive)

    // trigonometry
    cos(1.2); // 0.362... takes an angle *in radians*
    asin(0.3); // 0.304...returns value *in radians*
    tanh(3); // 0.995... hyperbolic tangent
\end{minted}


But we can still run into weird issues with numbers in PHP. As a general rule \textit{never trust floating point numbers}

\begin{minted}{php}
    floor((0.1 + 0.7) * 10); // 7 - oop!
\end{minted}


\section{Strings}

Strings are also very similar to JavaScript.
\\

You can use single- or double-quotes:

\begin{minted}{php}
    $firstName = "Casper";
    $lastName = 'Spoooky';
\end{minted}

Double-quotes allow you to interpolate values:

\begin{minted}{php}
    $fullName = "{$firstName} {$lastName}";
\end{minted}

We use curly-braces to enter interpolation mode and then use the variable name inside. The \texttt{\$} here is part of the variable, whereas in JavaScript interpolation it's part of the syntax for interpolation (so the dollar is \textit{outside} the curly-braces).
\\

Because interpolation is such a common thing to do, generally we'll use double-quotes for strings.

\subsection{Concatenation}

PHP uses \texttt{.} for concatenation, this avoids the issues that JavaScript had with the overloaded \texttt{+} operator. It can, however, lead to many a brain-fart as you try and use \texttt{+} when you mean \texttt{.}:

\begin{minted}{php}
    dump($firstName . $lastName); // "CasperSpooky"
    dump($firstName + $lastName); // PHP Warning - A non-numeric value encountered
    dump("1" + "2"); // 3 - coerces to numbers
\end{minted}

\subsection{String Functions}

Unlike in JS, strings are not objects in PHP. That means they don't have properties or methods. So we have to use functions to work with them:

\begin{minted}{php}
    strtolower("Blah"); // "blah"
    strtoupper("Blah"); // "BLAH"
    trim("   Blah  "); // "Blah"
    substr("Fishsticks", 4); // "sticks"
\end{minted}

There are many other string function in the \href{http://www.php.net/manual/en/ref.strings.php}{PHP documentation}.


\pagebreak


\section{Booleans}

As with JavaScript, PHP has the boolean values \texttt{true} and \texttt{false}. The only difference in PHP is that they're not case sensitive:

\begin{minted}{php}
    $bool = true;
    $bool = True; // also valid
    $bool = TRUE; // still valid
    $bool = TrUe; // totes valid
\end{minted}

For consistency it's best to stick with the lowercase version.

\subsection{Boolean Logic}

PHP has \texttt{\&\&}, \texttt{||}, and \texttt{!} which work in the same way as in JavaScript:

\begin{minted}{php}
    true && false; // false
    true || false; // true
    !true; // false
    !false; // true
\end{minted}

PHP also has the written versions \texttt{and} and \texttt{or}:

\begin{minted}{php}
    true and false; // false
    true or false; // true
\end{minted}

If you're coming from JavaScript, you should use the \texttt{\&\&} and \texttt{||} versions. The other versions have different ``precedence'' and will not always work as you expect.


\pagebreak


\subsection{Comparison Operators}

All your favourite comparison operators are back:
\\

\begin{small}
    \begin{tabularx}{\textwidth}{c l X}
        \textbf{Operator} & \textbf{Name} & \textbf{Description} \\
        \texttt{===} & strict equality & \texttt{true} if the values are the same \\
        \texttt{!==} & non-equality & \texttt{false} if the values are the same\\
        \texttt{<} & less than & \texttt{true} if the first value is less than the second value  \\
        \texttt{>} & greater than & \texttt{true} if the first value is greater than the second value\\
        \texttt{<=} & less than or equal to & \texttt{true} if the first value is less than or equal to the second value  \\
        \texttt{>=} & greater than or equal to & \texttt{true} if the first value is greater than or equal to the second value
    \end{tabularx}
\end{small}

\par\bigskip

As with JavaScript there are also the type-coercing \texttt{==} and \texttt{!=} operators, but these are best avoided.


\subsection{Falsy Values}

PHP has all the falsy values that JavaScript has. Empty arrays are also falsy in PHP (which they are \textit{not} in JavaScript):

\begin{itemize}
    \item \texttt{false} itself
    \item The number zero: \texttt{0}, \texttt{0.0}, \texttt{-0}, \texttt{-0.0}
    \item The empty string: \texttt{""}
    \item An empty array: \texttt{[]}
    \item \texttt{NULL}
\end{itemize}



\section{Additional Resources}

\begin{itemize}[leftmargin=*]
    \item \href{https://www.php.net/manual/en/language.operators.arithmetic.php}{PHP: Arithmetic Operators}
    \item \href{https://www.php.net/manual/en/ref.math.php}{PHP: Maths Functions}
    \item \href{http://php.net/manual/en/language.operators.logical.php}{PHP: Logical Operators}
    \item \href{http://php.net/manual/en/language.operators.comparison.php}{PHP: Comparison Operators}
\end{itemize}


\chapter{Control Flow}
Because they're both C-based languages, loops and conditionals in PHP and JavaScript are syntactically identical. Just remember that variables in PHP are not declared and always start with a \texttt{\$}.

\section{Conditionals}

\subsection{\texttt{if} Statements}

A basic \texttt{if} statement:

\begin{minted}{php}
    if ($x < 10) {
        // do a thing
    }
\end{minted}

With an \texttt{else}:

\begin{minted}{php}
    if ($x < 10) {
        // do a thing
    } else {
        // do the other thing
    }
\end{minted}


\pagebreak


An \texttt{else if}:

\begin{minted}{php}
    if ($x < 10) {
        // do a thing
    } else if ($x < 20) {
        // do this thing
    } else {
        // do the other thing
    }
\end{minted}

In PHP the space between \texttt{else} and \texttt{if} can be omitted:

\begin{minted}{php}
    if ($x < 10) {
        // do a thing
    } elseif ($x < 20) {
        // do this thing
    } else {
        // do the other thing
    }
\end{minted}

Although they're not technically necessary for single line blocks, you should \textit{always} use the curly braces around a block.


\subsection{Ternary Operator}

PHP also has the ternary operator. As with JavaScript, a ternary operator is an \textit{expression}:

\begin{minted}{php}
    // if $index is less than 0, set it to 5
    // otherwise decrement it
    $index = $index < 0 ? 5 : $index - 1;
\end{minted}


\pagebreak


\subsection{\texttt{switch} Statements}

\texttt{switch} statements are also available:

\begin{minted}{php}
    switch ($x) {
        case 1:
            $message = "It's One";
            break;
        case 2:
            $message = "It's Two";
            break;
        default:
            $message = "No idea";
    }
\end{minted}

Don't forget to \texttt{break} at the end of each \texttt{case}; and, remember, \texttt{switch} statements are only useful if you want to run different bits of code based on the \textit{same expression}.


\section{Loops}

\subsection{\texttt{for} Loops}

\texttt{for} loops are much the same as in JavaScript (except we don't declare the counter variable and there are dollars everywhere):

\begin{minted}{php}
    $total = 0;

    for ($i = 1; $i <= 10; $i += 1) {
        $total += $i;
    }

    dump($total); // 55
\end{minted}

Remember the three parts:

\begin{enumerate}
    \item Setup the counter variable (runs once before the loop starts)
    \item Loop condition: run the loop as long as this is \texttt{true}
    \item Evaluated after each iteration
\end{enumerate}

It's generally best to use a \texttt{for} loop if you know how many times the loop needs to run.

\subsection{\texttt{while} Loops}

While loops are useful if you don't know how many times the loop needs to run:

\begin{minted}{php}
    $i = 0;
    $total = 0;

    while ($total < 100) {
        $i += 1;
        $total += $i;
    }

    dump($total); // 105
\end{minted}

\subsection{\texttt{do-while} Loops}

A \texttt{do-while} loop is the same as a \texttt{while} loop, except that the \texttt{do} block will always run \textit{at least once}. They can sometimes be a little easier to work out:

\begin{minted}{php}
    $i = 0;
    $total = 0;

    do {
        $i += 1;
        $total += $i;
    } while ($total < 100);

    dump($total); // 105
\end{minted}

\hr

As with all loops, be careful not to create an infinite loop!


\pagebreak


\section{Additional Resources}

\begin{itemize}[leftmargin=*]
    \item \href{http://www.php.net/manual/en/control-structures.if.php}{PHP: \texttt{if}}
    \item \href{http://www.php.net/manual/en/control-structures.elseif.php}{PHP: \texttt{else if}}
    \item \href{https://www.php.net/manual/en/language.operators.comparison.php#language.operators.comparison.ternary}{PHP: The Ternary Operator}
    \item \href{http://www.php.net/manual/en/control-structures.switch.php}{PHP: \texttt{switch}}
    \item \href{http://www.php.net/manual/en/control-structures.for.php}{PHP: \texttt{for} Loops}
    \item \href{http://www.php.net/manual/en/control-structures.while.php}{PHP: \texttt{while} loops}
    \item \href{http://www.php.net/manual/en/control-structures.do.while.php}{PHP: \texttt{do-while} loops}
\end{itemize}


\chapter{Functions}
Functions in PHP serve the same purpose as functions in JavaScript: they allow us to reuse bits of code.
\\

They are written in the same way as functions were historically written in most C-style language (including JavaScript):

\begin{minted}{php}
    function add($a, $b) {
        return $a + $b;
    }

    $result = add(12, 34);
    dump($result); // int(46)
\end{minted}


\section{Scope}

Because we don't declare variables in PHP it takes a very explicit approach to scope: by default variables used inside functions are assumed to be \textit{locally} scoped\footnote{The \texttt{global} keyword allows us to get around this, but it's unlikely you'll ever need to}.

\section{Anonymous Functions}

Standard functions in PHP are not values like they are in JavaScript, so we can't just pass them round and assign them to variables in quite the same way. However, passing functions around is such a useful thing to be able to do that recent versions of PHP added \textbf{closures} as a way to do this:

\begin{minted}{php}
    $add = function ($a, $b) {
        return $a + $b;
    };

    $result = $add(1, 2);
    dump($result); // int(3)
\end{minted}

As you can see, because it's stored in a variable we need to use the \texttt{\$} when calling the function.
\\

Some functions in PHP require a \texttt{callable} argument, which just means a function:

\begin{minted}{php}
    $result = array_map(function ($value) {
        return $value * $value;
    }, [1, 2, 3, 4, 5]);

    dump($result); // [1, 4, 9, 16, 25]
\end{minted}

Closures don't have access to variables declared outside of themselves. In order to use these you use the \texttt{use} keyword:

\begin{minted}{php}
    $multiplyBy = 10;

    $result = array_map(function ($value) use ($multiplyBy) {
        return $value * $multiplyBy;
    }, [1, 2, 3]);

    dump($result); // [10, 20, 30]
\end{minted}

In PHP 7.4+ this could be written as:

\begin{minted}{php}
    $multiplyBy = 10;

    $result = array_map(fn($value) => $value * $multiplyBy, [1, 2, 3]);

    dump($result); // [10, 20, 30]
\end{minted}

This is PHP's new Arrow Function syntax. It avoids the need for \texttt{use} as it uses the parent scope. It also automatically returns a value, like with JS's fat arrow functions. However, it can \textbf{only be used if the function body is a single expression}: there's no way to have multiple lines.
\\

It's worth noting that in both cases you only get read access to \texttt{\$multiplyBy}: you cannot change its value inside the closure.



\section{Additional Resources}

\begin{itemize}[leftmargin=*]
    \item \href{http://www.php.net/manual/en/functions.user-defined.php}{PHP: Functions}
    \item \href{http://www.php.net/manual/en/language.types.callable.php}{PHP: Callable}
    \item \href{https://www.php.net/manual/en/functions.arrow.php}{PHP: Arrow Functions}
\end{itemize}


\chapter{Arrays}
Arrays come in two forms in PHP: numerically indexed and associative.
\\


\section{Numerically Indexed Arrays}

Numerically indexed arrays are exactly the same as in JavaScript.

\begin{minted}{php}
    $values = [1, 2, 3, 4, 5];
    $first = $values[0]; // first item in array
    $last = $values[4]; // last item in this array
    dump($first); // 1
    dump($last); // 5
\end{minted}

We can add a value to the end of an array:

\begin{minted}{php}
    $values[] = 6;
\end{minted}

We can also find out how many items are in an array:

\begin{minted}{php}
    count($values); // 5
\end{minted}

Arrays in PHP aren't objects like in JavaScript, so they don't have methods. We have to use built-in functions like \texttt{count}.


\begin{infobox}{Old Skool Arrays}
    If you're working with older PHP you may see arrays written using the older notation:

    \begin{minted}{php}
        $values = array(1, 2, 3, 4, 5);
    \end{minted}

    This isn't actually a function, although it does look like one.
    \\

    The newer notation was added in PHP 5.4, which has reached end-of-life, so you generally don't need to use the older notation. However, certain coding standards (such as WordPress) discourage the use of the newer style.
\end{infobox}

\section{Associative Arrays}

Associative arrays are effectively PHP's version of object literals: in PHP ``objects'' are (almost) always instances of a \texttt{class}, but the key-value pairing of object literals is still a useful concept:

\begin{minted}{php}
    $assoc = [
        "firstName" => "Ben",
        "lastName" => "Wales",
        "dob" => "2018-08-24",
    ];

    dump($assoc["lastName"]); // "Wales"
\end{minted}

Notice that the syntax is quite different from JS: square brackets to open, keys need quoting, and fat-arrow between the key and value.

\pagebreak

\begin{infobox}{All Arrays Are Associative}
    Technically speaking, all arrays in PHP are actually associative, they're just automatically given a numerical key if you don't provide one. This does mean you can run into some unusual results if you're not careful:

    \begin{minted}{php}
        $arr = [];
        $arr["key"] = 1; // key provided
        $arr[] = 2; // no key, so starts at 0

        dump($arr); // ["key" => 1, 0 => 2]
    \end{minted}

    You should not deliberately mix numerically indexed and associative arrays as things can get weird.
\end{infobox}


\section{Iterating Over Arrays}

We can use a \texttt{foreach} loop to iterate over every item in an array.

\begin{minted}{php}
    $values = [1, 2, 3, 4, 5];

    foreach ($values as $value) {
        // $value will be each value in turn
    }
\end{minted}

You can also get the key:

\begin{minted}{php}
    $assoc = [
        "firstName" => "Ben",
        "lastName" => "Wales",
        "dob" => "2018-08-24",
    ];

    foreach ($assoc as $key => $value) {
        // $key will be each key in turn
        // $value will be each value in turn
    }
\end{minted}

As numerically indexed arrays are just associative arrays with numerical keys, you can also use this syntax to get the index of a numerical array:

\begin{minted}{php}
    $values = [1, 2, 3, 4, 5];

    foreach ($values as $index => $value) {
        // $index will be each index in turn
        // $value will be each value in turn
    }
\end{minted}

We called it \texttt{\$key} in the first case and \texttt{\$index} in the second, but we can call it whatever we like, as long as it appears before the fat arrow.


\section{Array Iterator Functions}

PHP does have functions for doing \texttt{map}, \texttt{filter}, and \texttt{reduce} but they're inconsistent and not always usable (for example, if you need the current key/index).
\\

However, we can install Laravel's support package to gain access to ``Collections'', which makes working with arrays much nicer. It's got \href{http://laravel.com/docs/master/collections#available-methods}{tonnes of really useful methods}, but we'll just look at four of them here: our old friends \texttt{map()}, \texttt{filter()}, and \texttt{reduce()}, as well as a very useful one called \texttt{pluck()}.
\\

First we need to install the package:

\begin{minted}{bash}
    composer require illuminate/support
\end{minted}

Generally collection methods return a collection. You can turn a collection back into a standard array by calling its \texttt{all()} method.

\pagebreak

\subsubsection{\texttt{filter}}

Filter is almost identical to JavaScript: we pass it an anonymous function that takes each item in the array and returns a boolean value. It returns a new collection containing all the items for which the function returned \texttt{true}:

\php{}{05-arrays/figures/01-filter}

\subsubsection{\texttt{map}}

Map is also very similar to JavaScript: we pass it an anonymous function that takes each item in the array and transforms the value somehow. It returns a new collection where each item has been transformed:

\php{}{05-arrays/figures/02-map}

\pagebreak

\subsubsection{\texttt{reduce}}

Again, reduce is very similar to JavaScript: we pass it an anonymous function that takes the accumulated value and each item in the array. The return value is passed in as the accumulator value for the next iteration. It returns the final accumulated value:

\php{}{05-arrays/figures/03-reduce}

Make sure you pass in an initial value for the accumulator, otherwise it will be \texttt{null}, which might cause problems.

\subsubsection{\texttt{pluck}}

We've not come across \texttt{pluck} before, but it's very useful. It assumes your collection contains either associative arrays or objects all with the same structure. You pass it a key value and it extracts a new collection containing just that key/property from each item in the collection:

\php{}{05-arrays/figures/04-pluck}


\begin{infobox}{Under the Hood}
    The \texttt{collect} function is actually creating an instance of the \texttt{Collection} class, which wraps the array, and you're calling various methods on it. When you call the \texttt{all()} method it gives you back the array it's been working on.
\end{infobox}


\section{Additional Resources}

\begin{itemize}[leftmargin=*]
    \item \href{http://www.php.net/manual/en/language.types.array.php}{PHP: Arrays}
    \item \href{https://www.php.net/manual/en/ref.array.php}{PHP: Array Functions}
    \item \href{http://www.php.net/manual/en/control-structures.foreach.php}{PHP: \texttt{foreach}}
    \item \href{http://laravel.com/docs/master/collections}{Collections}
\end{itemize}


\chapter{Regular Expressions}
Regular Expressions (or ``regex'' for short) are a way to check/search a string for a \textit{pattern} as opposed to a specific string.
\\

They're a little bit mad looking to start with. In fact they're a little bit mad looking even after you've been using them for years. But they are very useful as long as you don't get too carried away.
\\

For example, we might want to split a string on a comma followed by \textit{any} number of spaces (not just one). We'd write that using the following regex:

\begin{minted}{text}
    /,\s*/
\end{minted}


Or if we wanted to match any combination of lowercase letters, numbers, underscores, and hyphens between 3 and 16 characters long we could write this with the following regex:

\begin{minted}{text}
    /^[a-z0-9_-]{3,16}$/
\end{minted}

JavaScript also supports regexes (it's actually its own \textit{type} in JS, like numbers and strings).

\pagebreak


\section{Parts}

We're not going to get into every aspect of Regexes, but we'll cover enough for them to be useful.

\subsection{Literal Strings}

The first thing to be aware of is that a RegEx of just non-special characters represents that string. So \texttt{wombat} is just the string ``wombat''.

\subsection{Quantifiers}

Quantifiers allow us to specify that a character should appear zero, one, or many times.

\begin{center}
    \begin{small}
        \begin{tabularx}{\textwidth}{r l}
            \textbf{Quantifier}   & \textbf{Description} \\
            \texttt{*}          & 0 or more \\
            \texttt{+}          & 1 or more \\
            \texttt{?}          & 0 or 1 \\
            \texttt{\{3\}}      & exactly 3 \\
            \texttt{\{3,\}}     & 3 or more \\
            \texttt{\{3,5\}}   & 3, 4, or 5
        \end{tabularx}
    \end{small}
\end{center}

For example:

\begin{minted}{text}
    /a+/        - would match 'a', 'aa', 'aaa', 'aaaa', etc.
    /b*/        - would match '', 'b', 'bb', 'bbb', etc.
    /c?/        - would match '' and 'c'
    /d{3}/      - would match 'ddd'
    /d{3,5}/    - would match 'ddd', 'dddd', and 'ddddd'
    /abc*/      - would match 'ab', 'abc', 'abcc', 'abccc', etc.
    /(abc)*/    - would match '', 'abc', 'abcabc', 'abcabcabc', etc.
    /https?:/   - would match 'http:' and 'https:'
\end{minted}

Notice that we can use brackets to quantify more than one character.

\pagebreak

\subsection{Groups \& Ranges}

Groups and ranges allow us to specify a range of characters that we're interested in.

\begin{center}
    \begin{small}
        \begin{tabularx}{\textwidth}{r l}
            \textbf{Range}          & \textbf{Description} \\
            \texttt{[a-z]}          & all lowercase letters \\
            \texttt{[A-Z]}          & all uppercase letters \\
            \texttt{[0-9]}          & all numbers \\
            \texttt{[abc]}          & 'a', 'b', or 'c' \\
            \texttt{[0-9afg]}       & all numbers plus 'a', 'f', and 'g' \\
            \texttt{[a-zA-Z]}       & all letters, case-insensitive \\
            \texttt{[a-zA-Z0-9]}    & alphanumeric characters \\
            \texttt{[0-9A-F]}       & valid hexadecimal digits \\
            \texttt{[\textasciicircum abc]}         & \textbf{not} 'a', 'b', or 'c'
        \end{tabularx}
    \end{small}
\end{center}

A group is a set of characters wrapped with square brackets that will match \textit{any} character in the group.
\\

A range, which must always be used \textit{inside} a group, represents a series of characters, e.g. all the letters between \texttt{a} and \texttt{z}.
\\

For example:

\begin{minted}{text}
    /[a-z]+/      - any number of lowercase letters
    /[a-z0-9_-]/  - a single lowercase letter, digit, underscore, or hyphen
\end{minted}

\subsection{Special Characters}

These represent special characters like tab and a new line:

\begin{center}
    \begin{small}
        \begin{tabularx}{\textwidth}{r l}
            \textbf{Character}          & \textbf{Description} \\
            \texttt{\textbackslash n}          & a new line \\
            \texttt{\textbackslash r}          & a carriage return \\
            \texttt{\textbackslash t}          & a tab \\
        \end{tabularx}
    \end{small}
\end{center}

\texttt{\textbackslash r} tends to only crop up if you read directly from a file generated on a Windows machine.
\\

These can generally be used in regular strings too.

\pagebreak

\subsection{Character Classes}

Character classes are shortcuts for specific ranges.

\begin{center}
    \begin{small}
        \begin{tabularx}{\textwidth}{r l}
            \textbf{Class}                    & \textbf{Description} \\
            \texttt{\textbackslash s}         & white\textbf{s}pace (spaces, tabs, etc.) \\
            \texttt{\textbackslash S}         & \textit{not} white\textbf{s}pace \\
            \texttt{\textbackslash w}         & \textbf{w}ord (\texttt{[A-Za-z0-9\_]}) \\
            \texttt{\textbackslash W}         & \textit{not} \textbf{w}ord \\
            \texttt{\textbackslash d}         & \textbf{d}igit \\
            \texttt{\textbackslash D}         & \textit{not} \textbf{d}igit
        \end{tabularx}
    \end{small}
\end{center}

For example:

\begin{minted}{text}
    /,\s*/      - a comma followed by 0 or more whitespace characters
    /\w\s+\w/   - two words separated by 1 or more whitespace characters
\end{minted}


\subsection{Dot}

In a regex the \texttt{.} character has a special meaning: \textit{any} character except for \texttt{\textbackslash n}. You need to be careful using it, particularly with the \texttt{*} and \texttt{+} quantifiers.

\begin{minted}{text}
    /.+@.+/     - '@' symbol with something either side
\end{minted}

If you want to match an actual full stop you need to ``escape'' it with a backslash:

\begin{minted}{text}
    /\.+@\.+/   - an '@' symbol with some number of '.' either side
\end{minted}



\subsection{Anchors}

Sometimes \textit{where} the substring appears is important.

\begin{center}
    \begin{small}
        \begin{tabularx}{\textwidth}{r l}
            \textbf{Anchor}             & \textbf{Description} \\
            \texttt{\textasciicircum}   & beginning of the string \\
            \texttt{\$}                 & end of the string \\
        \end{tabularx}
    \end{small}
\end{center}

For example:

\begin{minted}{text}
    /^abc/          - would match 'abc' but not '0abc'
    /abc$/          - would match 'abc' but not 'abc0'
\end{minted}


\section{Regex with PHP}

We can use regexes for all sorts of string manipulations. The three most common are:

\begin{itemize}
    \item Searching a string
    \item Splitting a string
    \item Replacing a string
\end{itemize}


\subsection{\texttt{preg\_match}}

The \texttt{preg\_match} function can be used to check if a string matches a regular expression:

\begin{minted}{php}
    // does the string contain one or more 'l' characters
    preg_match("/l+/", "Hello"); // 1

    // does the string *start* with one or more 'l' characters
    preg_match("/^l+/", "Hello"); // 0

    // does the string contain two words, separated by a space
    preg_match("/\w\s+\w/", "Hello There World"); // 1

    // does the string consist of *just* two words, separated by a space
    preg_match("/^\w\s+\w$/", "Hello There World"); // 0
    preg_match("/^\w\s+\w$/", "Hello Mum"); // 1
\end{minted}

It returns \texttt{1} if a match is found and \texttt{0} if it is not. Make sure you always use \texttt{===} when checking the result, as it returns \texttt{false} if an error occurs – which might get confused for \texttt{0} if you use \texttt{==}:

\begin{minted}{php}
    if (preg_match("/l+/", "Hello") === 1) {
        // matches one or more 'l' characters
    }
\end{minted}

\pagebreak

\subsection{\texttt{preg\_split}}

\texttt{preg\_split} can be used to split a string on a certain regex:

\begin{minted}{php}
    $csv = "first, second,   third,fourth";

    // split on a comma followed by 0 or more spaces
    $result = preg_split("/,\s*/", $csv);

    // [
    //    [0] => "first",
    //    [1] => "second",
    //    [2] => "third",
    //    [3] => "fourth"
    // ]
\end{minted}

We pass it a regex and a string and it gives us back an array of strings where the original string has been split on the regex.

\subsection{\texttt{preg\_replace}}

The \texttt{preg\_replace} function can be used to replace part of a string that matches a regex with something else:

\begin{minted}{php}
    $str = 'blah      blah   blah';

    // replace one or more space with a single space
    $tidied = preg_replace("/\s+/", " ", $str);

    // "blah blah blah"
\end{minted}


There is a lot more to \texttt{preg\_replace} than this basic example,\footnote{``Capture groups'' and ``back references'' are particularly useful} but we'll keep it simple for now.

\pagebreak

\begin{infobox}{Flags}
    In PHP we can add various \textbf{flags} to the end of the regex. These go after the last forward-slash, e.g. \texttt{"/[a-z]*/i"}.
    \\

    There are three particularly useful ones in PHP:
    \\

    \begin{tabularx}{\textwidth}{r l X}
        \textbf{Flag}    & \textbf{Name}    & \textbf{Description} \\
        \texttt{i}       & case insensitive & pattern will match upper and lower case \\
        \texttt{m}       & multi-line       & separate lines count as separate strings for anchors\\
        \texttt{s}       & dot all          & the \texttt{.} character should include new lines\\
    \end{tabularx}
\end{infobox}


\section{Alternatives to Regex}

For basic validation you are often better using PHP's \texttt{filter\_var} function:

\begin{minted}{php}
    $email = "penny@hello.horse";
    $valid = filter_var($email, FILTER_VALIDATE_EMAIL);

    if ($valid) {
        // valid email address
    }
\end{minted}

The \texttt{filter\_var} function takes a string and a filter type. It then returns the filtered string if it is valid or \texttt{false} otherwise.
\\

Here are some particularly useful filters:\footnote{These big shouty looking things are constant variables defined by PHP – in C-based languages constants are often written in uppercase with underscores}

\begin{itemize}
    \item \texttt{FILTER\_VALIDATE\_EMAIL}
    \item \texttt{FILTER\_VALIDATE\_DOMAIN}
    \item \texttt{FILTER\_VALIDATE\_URL}
\end{itemize}

There's a full list \href{https://www.php.net/manual/en/filter.filters.validate.php}{on the PHP docs}.



\section{The Dangers of Regex}

\quoteinline{Some people, when confronted with a problem, think ``I know, I'll use regular expressions.'' Now they have two problems.}{Jamie Zawinski}

It's not uncommon for people new to programming to try and solve complex string manipulations using regular expressions. This can lead to hard to read and inefficient code. There are many problems that require a \textbf{parser}: a much more clever sort of algorithm that can elegantly cope with things like matching start/end tags.
\\

As a general rule, if your regular expression isn't easy to understand in one glance, then you probably shouldn't be using it.
\\

Fine:

\begin{minted}{text}
    /,\s*/
\end{minted}

Fuck off:

\begin{minted}[fontsize=\tiny]{text}
    /^(?:(?:https?|ftp)://)(?:\S+(?::\S*)?@|\d{1,3}(?:\.\d{1,3}){3}|(?:(?:[a-z\d\x{00a1}-\x{ffff}]+-?)*[a-z\d\x{00a1}-\x{ffff}]+)(?:\.(?:[a-z\d\x{00a1}-\x{ffff}]+-?)*[a-z\d\x{00a1}-\x{ffff}]+)*(?:\.[a-z\x{00a1}-\x{ffff}]{2,6}))(?::\d+)?(?:[^\s]*)?$/iu
\end{minted}



\section{Additional Resources}

\begin{itemize}[leftmargin=*]
    \item \href{https://regexr.com}{Regexr}: an online Regex testing tool - make sure you set it to use ``PCRE''
    \item \href{http://www.php.net/manual/en/function.preg-match.php}{PHP: \texttt{preg\_match}}
    \item \href{http://www.php.net/manual/en/function.preg-match-all.php}{PHP: \texttt{preg\_matchall}}
    \item \href{http://www.php.net/manual/en/function.preg-replace.php}{PHP: \texttt{preg\_replace}}
    \item \href{http://www.php.net/manual/en/function.preg-split.php}{PHP: \texttt{preg\_split}}
    \item \href{https://www.freecodecamp.org/learn/javascript-algorithms-and-data-structures/regular-expressions/using-the-test-method}{freeCodeCamp: RegEx Exercises with JavaScript}
    \item \href{https://blog.codinghorror.com/regular-expressions-now-you-have-two-problems/}{Regular Expressions: Now You Have Two Problems}
    \item \href{http://www.regexcrossword.com/}{RegEx Crossword}
    \item \href{https://stackoverflow.com/a/1732454}{Stack Overflow: RegEx match open tags except XHTML self-contained tags}
\end{itemize}


\chapter{Classes}
In PHP (and all object-oriented languages), an \textbf{object} is a way of storing some values (\textbf{properties}) and functionality (\textbf{methods}) that are related in some way.
\\

It's very common to need many objects that have the same methods and property names, but where the \textit{values} of the properties are specific to each object.
\\

For example we might have two objects, that both represent people: both have a \texttt{firstName} property, a \texttt{lastName} property, and a \texttt{birthdate} property as well as a \texttt{getAge()} method, which returns their age based on the date of birth and the current date. The property names and methods will be identical for both objects, however the \textit{values} for the properties will be different: a ``Mark'' object would have \texttt{"Mark"}, \texttt{"Wales"}, and \texttt{"1984-04-16"} for the respective values, whereas a ``Ben'' object would have \texttt{"Ben"}, \texttt{"Wales"}, and \texttt{"2018-08-24"}.
\\

\img{25em}{07-classes/img/objects}{1.2em}{Two different objects with the same properties, but different values}

\pagebreak

A \textbf{class} is an abstract representation of an object that you want to create. For example, you might have a class \texttt{Person} that allows you to create lots of object \textbf{instances} representing different people.
\\

Here's a class that represents a person:

\php{}{07-classes/figures/01-Person}

Here's how we'd use our class:

\php{}{07-classes/figures/02-usage}

You can see that where we'd write a dot in JavaScript (\texttt{ben.getAge()}), we write an arrow in PHP (\texttt{\$ben->getAge()}), but otherwise it's almost identical in usage.

\begin{infobox}{PSR-2: Coding Style Guide}
    You've possibly noticed that in all the examples above the opening curly brace (\texttt{\{}) for classes and methods is on its own line. This is part of the \href{https://www.php-fig.org/psr/psr-2/}{PSR-2: Coding Style Guide} spec.
    \\

    If you do an \texttt{if} statement (or other control structure) then the opening curly brace, obviously, goes on \textit{the same} line.
    \\

    You're probably thinking that this doesn't make the slightest bit of sense. And you'd be right. PSR-2 was created by sending round a questionnaire about coding style to 30 or so of the most prolific PHP programmers and they just went with whatever the majority said for each point.
    \\

    But it's the style that everyone uses now. You'll get used to it.
\end{infobox}

\section{Properties}

In most OOP languages we can declare the properties that our object has\footnote{JavaScript is still relatively immature as a ``classical'' OOP language, but support for this is coming}. We normally declare all of our properties at the top of the class: this makes it easy to see in one glance what properties the object instance will have.

\php{}{07-classes/figures/03-properties}

We declare properties by using the \texttt{private}\footnote{More on this later} keyword followed by the name of the property – with a dollar at the front, just like with variables. However, properties are \texttt{not} variables, as they belong to a specific object instance: each object instance with its own set of values.

\subsection{Default Values}

We can assign properties default values when we declare them. For example, we could add an \texttt{\$arms} property to the \texttt{Person} class and set it to \texttt{2} by default:

\php{}{07-classes/figures/04-property-values}

You can always use a method to change the value later.
\\

If you declare a property but don't assign it a value it will have the value \texttt{null}.


\section{\texttt{\$this}}

Inside our classes we can use the \texttt{\$this} keyword to access properties and methods that belong to the current object instance. It works in much the same way as JavaScript except that it's much more reliable: \texttt{\$this} in PHP \textit{always} refers to the current object and has no meaning elsewhere.

\php{}{07-classes/figures/05-Address}

Notice that it's the \texttt{\$this} bit that has the dollar in front of it, the property name does not, even though they \textit{do} have a dollar in front of them when we declare them at the top of the class\footnote{For weird historical reasons}.

\section{Constructor}

When we create a new object instance with \texttt{new} any values that we pass as arguments will be given to the \texttt{\_\_construct()} method, just as if we'd called it ourselves. So the number of arguments that we pass with \texttt{new} should always match the number of parameters that the \texttt{\_\_construct()} method accepts:

\php{}{07-classes/figures/06-construct}

We need to make sure we store the values passed in as properties on the newly created instance, otherwise they'll cease to exist once the \texttt{\_\_construct()} method has completed\footnote{Remember, parameters only exist inside the function they belong to}. It's fairly standard to name the parameters the same thing as the properties that they'll be assigned to as in the above example.
\\

If we've given a property a default value we don't need to set it up in \texttt{\_\_construct()} as well, it will automatically be given the default value when the object instance is created. If \textit{all} of your properties have default values then a \texttt{\_\_construct()} method might not even be necessary\footnote{It's not uncommon in full OOP code to accept no parameters in the constructor and to use setters for everything.}.

\pagebreak

\begin{infobox}{Using \texttt{new} Objects}
    Unlike in JavaScript you can't immediately use a created object:

    \begin{minted}{php}
        // won't work
        new Person("Jim", "Henson", "1936-09-24")->getAge();
    \end{minted}

    If you really want to, you can get around this with a pair of brackets around the \texttt{new} statement:

    \begin{minted}[frame=topline]{php}
        // will work
        // but you don't have a reference to the object anymore
        (new Person("Jim", "Henson", "1936-09-24"))->getAge();
    \end{minted}

    However, this isn't normally necessary.
\end{infobox}


\section{Additional Resources}

\begin{itemize}[leftmargin=*]
    \item \href{https://phpapprentice.com/classes.html}{PHP Apprentice: Classes}
    \item \href{https://laracasts.com/series/php-for-beginners/episodes/12}{Laracasts: Classes 101}
    \item \href{https://laracasts.com/series/object-oriented-bootcamp-in-php/episodes/1}{Laracasts: Classes}
\end{itemize}


\begin{readonly}
    \chapter{Static Methods \& Properties}
    Classes are primarily used for creating object instances. But sometimes it's useful to write some functionality about the object type instead of object instances.
\\

For example, if we have a \texttt{Person} class we might want to write a bit of functionality that given an array of \texttt{Person} objects gives us back an array of just last names. We could write a \texttt{lastNames()} function in global scope, but then it's not associated with the \texttt{Person} class.
\\

Instead, we will write a \texttt{static} method: a method that belongs to the class itself rather than to an object instance.

\php{}{08-static/figures/01-static}

Now it is clear that the \texttt{lastNames()} method has something to do with \texttt{Person} objects.


\begin{infobox}{Paamayim Nekudotayim}
    The \texttt{::} symbol is also known as the ``Paamayim Nekudotayim'', which is Hebrew for ``double colon''. This can lead to the somewhat mystifying error:

    \begin{minted}{diff}
        PHP expects T_PAAMAYIM_NEKUDOTAYIM
    \end{minted}

    All it's saying is you need a \texttt{::} somewhere.
    \\

    The \href{https://en.wikipedia.org/wiki/Zend_Engine}{Zend Engine}, which was behind PHP 3.0 and all subsequent releases, was originally developed at the Israel Institute of Technology.
\end{infobox}

\section{The \texttt{static} Keyword}

Sometimes it's useful to be able to refer to the class you're currently working in. We can use the \texttt{static} keyword to do this.
\\

For example, we might want a separate \texttt{static} method that returns the last names alphabetically – but it would be repetitive to rewrite the method we've already got:

\php{}{08-static/figures/02-keyword}

\begin{infobox}{\texttt{static} vs \texttt{self}}
    You will sometimes see \texttt{self} instead of \texttt{static} to reference the current class. Using \texttt{static} in this way was only added in PHP 5.3, so a lot of older code uses \texttt{self}.
    \\

    If you're not using inheritance, then it doesn't make any difference which one you use. If you do then \texttt{self} refers to class that it is written in and \texttt{static} refers to the class it is called in (which might be different from where it was written if you're using inheritance).
\end{infobox}


\section{Static Properties}

You can also have \texttt{static} properties. Because they belong to the class they exist for as long as your code is running, so you can use them for storing values that you want to keep around – like global variables, but with a dedicated home. This is very useful for caching values.
\\

Say we needed to create a \texttt{\$renderer} object that all our object instances can use to render\textellipsis{} something. We could use a \texttt{static} property to store it, so that we only create it one time:

\php{}{08-static/figures/03-properties}


\section{Additional Resources}

\begin{itemize}[leftmargin=*]
    \item \href{https://phpapprentice.com/static.html}{PHP Apprentice: Static}
\end{itemize}

\end{readonly}

\chapter{Object-Oriented Programming}
So far, all of the PHP code\footnote{And most of the JavaScript too – except for event handlers} you've written has been ``procedural'': start at the top of a file, run through it, maybe call a few functions as you go, and then finish at the end. This is fine for simple programs or when we're just working inside an existing system (e.g. WordPress), but it doesn't really scale to larger applications.
\\

The problem comes because we need to manage \textbf{state}: keeping track of all the values in our code. For a large app you could easily have thousands of values that need storing. Naming and keeping track of all these variables would become a nightmare if they were all in the same global scope.
\\

\textbf{Object-Oriented Programming}\footnote{``Oriented'' not ``Orientated''.} (OOP) is one way to make this easier. We keep functions and variables that are related to each other in an object. We then get different objects to talk to one another, so that rather than having lots of global variables and functions, all of our data lives inside relevant objects. This is a very different way of thinking about code, but can be very elegant if you can wrap your head around some of the intricacies we'll be covering over the next few days.

\pagebreak

\begin{infobox}{The Unusual History of PHP}
    PHP has a long and complicated history. The first version of PHP wasn't even a programming language, it was just simple templating language that allowed you to re-use the same HTML code in multiple files.
    \\

    Over the years PHP morphed into a simple programming language and then into a modern object-oriented programming language. However, it wasn't really until 2009, with the release of PHP 5.3, that PHP could truly be considered a fully object-oriented language.
    \\

    Because of this gradual change, older PHP frameworks and systems (such as WordPress) were originally written using non-OO code, which is why they still contain a large amount of procedural code.
    \\

    PHP gets a lot of flack for not being a very good programming language and a few years ago that was perhaps a valid criticism. But in recent years, particularly with the release of PHP 7, it's just not true anymore. It certainly still has some issues, but nothing that a few libraries can't get around.

    \quoteinline{There are only two kinds of languages: the ones people complain about and the ones nobody uses}{Bjarne Stroustrup, Creator of C++}
\end{infobox}

\pagebreak

\section{Objectification}

Say that our app includes some code to send an email. If we were using procedural code we would probably have a function called \texttt{sendMail} that we can pass various values to:

\php{}{09-oop/figures/01-mail}

But we might want to be able to customise more than just the to, from, and message parts of the email. Which means we'd either need to have a lot of optional arguments (which becomes unwieldy quickly) or rely on global variables:

\php{}{09-oop/figures/02-aaargh}

But this is horrible: we have no way of preventing other parts of our code from changing these values and we would start having to use long variable names to avoid ambiguity in bigger apps.
\\

So, we want to store the variables and the functionality together in one place and in such a way that values can't be accidentally changed. This is where objects come in:

\php{}{09-oop/figures/03-Mail}

Now if we need to add additional fields, we can just add a property and setter method:

\php{}{09-oop/figures/04-extra}

Now we can add an array of \texttt{bcc} addresses. And because we've set it to an empty array by default it wouldn't matter if we didn't set any – the code should still work as before.
\\

This is a much nicer way of working with code.


\section{Chaining}

You'll notice that all the methods that \textit{set} a value return \texttt{\$this}. As a general rule of thumb, if a method doesn't have anything sensible to return, for example if it just sets a value, then you can return \texttt{\$this}. \texttt{\$this} represents the current object instance, which allows you to \textbf{chain} methods together:

\begin{minted}{php}
    $mail->to("bob@bob.com")->from("hello@wombat.io")
         ->bcc([
           "ada@lovelace.dev",
           "donald@knuth.horse",
         ])->send(
           "A Wombat Welcome",
           "Welcome to the best app for finding wombats near you"
         );
\end{minted}

If we don't return anything from a method we get \texttt{null} back, which isn't very useful:

\begin{minted}{php}
    $result = null;
    $result->doThing(); // PHP Fatal error: Call to a member function doThing() on null
\end{minted}

By returning \texttt{\$this} we get back the object instance (in the above case, the thing stored in \texttt{\$mail}) each time, meaning we can call another method straight away.
\\

If the methods hadn't returned \texttt{\$this} we'd have had to write it using the \texttt{\$mail} variable each time:

\begin{minted}{php}
    $mail->to("bob@bob.com");
    $mail->from("hello@wombat.io");
    $mail->bcc([
        "ada@lovelace.dev",
        "donald@knuth.horse",
    ]);
    $mail->send(
        "A Wombat Welcome",
        "Welcome to the best app for finding wombats near you"
    );
\end{minted}



\section{Object-to-Object}

In OOP objects use other objects to get things done. The key skill of OOP is getting the right objects to talk to one another.
\\

For example, say that you wanted to send some emails to a mailing list. Rather than having the mailing list object do all of the work, we could instead have one object that deals with keeping track of who is in the mailing list and another that deals with sending the email.
\\

For example, we could create a \texttt{Mail} class that just deals with creating and sending an email:

\php{}{09-oop/figures/05-Mail}

It has various methods for adding the necessary email fields and another for actually sending the mail (well, pretending to).
\\

The mailing list class can be relatively simple too – it just keeps track of all the subscribers and has a method for sending an email to each of them:

\php{}{09-oop/figures/06-MailingList}

But notice that the actual sending of the mail is done by calling methods on a \texttt{Mail} object that has been passed in.
\\

We could write something like the following to get it working together:

\php{}{09-oop/figures/07-app}

By passing the \texttt{Mail} code \textit{into} the mailing list class we can \textbf{compose} their behaviour to create a more complex behaviour\footnote{We could probably split up the \texttt{Mail} class up even further so that the sending code was separate}.


\section{The Law of Demeter}

The ``Law of Demeter'' is a guideline for OOP about how objects should use other objects. Expressed succinctly:
\\

\begin{center}
    \textit{Each object should only talk to its friends; don't talk to strangers}
\end{center}
\par\bigskip


In practice, this means that an object should only call methods on either itself or objects that it has been given/created. You should avoid calling a method which returns an object and then calling a method on that object: it requires too much knowledge about other objects.

\php{}{09-oop/figures/08-demeter}

You should be careful using chaining (returning \texttt{\$this} from a method). If the method returns a \textit{different} type of object then it's easy to break the Law of Demeter without realising it.



\section{Additional Resources}

\begin{itemize}[leftmargin=*]
    \item \href{ https://phpenthusiast.com/object-oriented-php-tutorials/chaining-methods-and-properties}{PHP Enthusiast: Chaining Methods and Properties}
    \item \href{https://en.wikipedia.org/wiki/Law_of_Demeter}{Wikipedia: The Law of Demeter}
\end{itemize}



\nchapter{Glossary}
\begin{itemize}[leftmargin=*]
    \item
        \textbf{Class}:
        An abstract representation of an object instance
    \item
        \textbf{Composer}
        PHP's package management system
    \item
        \textbf{Dependency Injection}
        Rather than hard-coding a dependency on a specific class, taking advantage of polymorphism and passing in a class that implements a specific interface.
    \item
        \textbf{Encapsulation}
        Keeping the inner-workings of a class private so that they can only be accessed by sending appropriate messages (by calling methods)
    \item
        \textbf{Inheritance}
        A way to share code between different classes. Should be used with caution as it breaks aspects of encapsulation.
    \item
        \textbf{Instance}:
        An object is an instance of a specific class with its own set of properties
    \item
        \textbf{Interface}:
        A list of method signatures that tell us how to talk to an object that implements it
    \item
        \textbf{Namespace}:
        A set of classes where each class has a unique name. We can have two classes with the same name as long as they are in different namespaces.
    \item
        \textbf{Polymorphism}:
        When two different classes can be used interchangeably in a specific context because they share the relevant method signatures
    \item
        \textbf{Single Responsibility Principle}
        The principle that an object/class should only do one sort of thing. We get more complex behaviour by composing different objects – similar to function composition in functional programming languages.
    \item
        \textbf{Static}:
        A property or method that belongs to a class rather than an object instance
\end{itemize}



\input{../../latex-templates/colophon.tex}

\end{document}
