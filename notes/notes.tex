% setup the document
\documentclass[b5paper,openany]{book}

% setup variables ------------------------------

% your name
\newcommand\instructor{Mark Wales \& Oli Ward}

% the week number
\newcommand\weekno{6}

% week main title
\newcommand\maintitle{Object-Oriented Programming}

% week subtitle
\newcommand\subtitle{with PHP}

% link to notes - relative to github.com/develop-me/
\newcommand\github{week-06--php/blob/master/notes/}

% setup functions, styling, etc.
\input{../../templates/template.tex}

% the structure of the document
\begin{document}

% render the title page (uses variables above)
\tp

\quotepage{Certainly not every good program is object-oriented, and not every object-oriented program is good.}{Bjarne Stroustrup, Creator of C++}

\tableofcontents

\input{../../templates/preface.tex}

\chapter{PHP Basics}
PHP is a general purpose programming language that is frequently used to run the \textbf{server-side} code for websites. It was originally created as a simple \textbf{templating language}, based loosely on Perl, to allow outputting HTML with repeated elements. Over the years it has evolved into a fully \textbf{object-oriented programming language}.
\\

PHP should look quite familiar if you've done JavaScript as they are both ``C-based'' languages, meaning that they share a syntax style: semi-colons, curly braces, and brackets.
\\

For now we're just going to run PHP in the command-line like we did in Week 3 with JavaScript. PHP doesn't have a REPL built-in like Node, so we always need to run a file:

\begin{minted}{bash}
    php file.php # run the given file
\end{minted}


\section{Hello, World!}

As is tradition, we should write a ``Hello, World!'' program before moving forward:

\begin{minted}{php}
    <?php

    echo "Hello, World!";
\end{minted}

We ``echo'' the string ``Hello, World!'', this is PHP's equivalent of using \texttt{console.log()} - although it's only useful when used with strings. \texttt{echo} is \textit{not} a function (or method), but instead a \textbf{keyword}. All that really means is that you don't \textit{call} it.
\\

As you can see, it's not that different from JavaScript. The main difference is that on line 1 we have to tell PHP that it's PHP. If we don't do this then PHP has a bit of an identity crisis and doesn't know what's going on.
\\

\textbf{Make sure you never have anything before the opening \texttt{<?php} tag - no spaces, line breaks, or comments - as this will break most modern PHP apps.}


\begin{infobox}{Ye Olde PHP}
    The opening \texttt{<?php} tag can seem a bit strange in modern PHP, as it doesn't seem to serve any purpose - surely it knows it's PHP?
    \\

    When PHP was used mainly as a templating language PHP files would be made up mostly of HTML with only snippets of PHP:

    \begin{minted}{html}
        <body>
            <h1><?php echo $header ?></h1>
            <div>
                <?php echo $content ?>
            </div>
        </body>
    \end{minted}

    PHP is still used like this in some PHP frameworks such as WordPress. Nowadays templating languages like Blade and Twig are more commonly used, thus separating the program logic from the templating language.
\end{infobox}

In JavaScript you could get away without semi-colons at the end of lines. PHP isn't nearly so forgiving: if you forget a semi-colon you will get a syntax error and the code will refuse to run.
\\

\textbf{From now on code examples won't include the opening tag, but you will need to add it as the first line in all your files.}


\section{\texttt{var\_dump}}

As mentioned above \texttt{echo} isn't that useful for values that aren't strings, so we'll be using \texttt{var\_dump} instead. This outputs not only the value we're interested but also relevant information about it.
\\

For example, running the following:

\begin{minted}{php}
    var_dump("Hello");
    var_dump(12 + 12);
    var_dump(true);
\end{minted}

Would give us:

\begin{minted}{text}
    string(5) "Hello"
    int(24)
    bool(true)
\end{minted}

\texttt{var\_dump} \textit{is} a function, so we call it and pass it values.
\\

\texttt{var\_dump} isn't brilliant for working with complex values, so we'll switch to using a nicer way of doing it once we've learned about using PHP libraries.


\section{Variables}

In PHP we don't \textit{declare} variables, we just start using them. This is possible because variables have to start with a \texttt{\$} (this is because of PHP's Perl influence).
\\

This makes it much easier to accidentally change the values of existing variables, as there is no difference between creating a new variable and reassigning an existing one. So be very careful when naming things.

\begin{minted}{php}
    $name = "Archie";

    $age = 4;
    $houseNumber = 21;

    $name = "Ben"; // changes the value of $name - deliberate?

    // using variables
    $notUseful = $age + $houseNumber; // 25
\end{minted}

Variables must start with a dollar, followed by a letter or underscore, followed by any number of letters, numbers, or underscores. Variable names are case-sensitive.
\\

As with JavaScript it is a standard convention to use \texttt{\$camelCase} for variable names in PHP, although you may see \texttt{\$snake\_case} used in older code.


\begin{infobox}{Documentation: A Warning}
    Be careful using the official PHP documentation: \textit{anyone} can submit ``User Contributed Notes'' and they often contain code samples that are to be avoided. This is often because the notes were added years ago when PHP was a very different language.
    \\

    As a general rule, don't look at the ``User Contributed Notes'' section: use Stack Overflow if the documentation hasn't cleared it up for you.
\end{infobox}



\section{Additional Resources}

\begin{itemize}[leftmargin=*]
    \item \href{https://www.php.net/manual/en/function.var-dump.php}{PHP: \texttt{var\_dump}}
    \item \href{http://php.net/manual/en/language.variables.basics.php}{PHP: Variable Basics}
    \item \href{https://en.wikipedia.org/wiki/PHP}{Wikipedia: PHP}
    \item \href{https://en.wikipedia.org/wiki/Perl}{Wikipedia: Perl}
\end{itemize}


\chapter{Basic Types}
PHP has the same basic types as JavaScript: numbers, strings, and booleans. They work in much the same way.


\section{Numbers}

Unlike JavaScript PHP does have a sense of whether a number is an integer (whole number) or a ``floating point number'' (one with a decimal place). This means that some of the issues we had with JavaScript and decimal numbers don't cause issues in PHP:

\begin{minted}{php}
    var_dump(12 + 12); // int(24)
    var_dump(0.1 + 0.2); // float(0.3) - hurrah!
\end{minted}


The operators should be familiar:
\\

\begin{small}
    \begin{tabu}{c l X}
        \textbf{Operator} & \textbf{Name} & \textbf{Description} \\
        \texttt{+}  & addition        & adds two numbers together \\
        \texttt{-}  & subtraction     & subtracts the second number from the first number \\
        \texttt{*}  & multiplication  & multiplies two numbers \\
        \texttt{/}  & division        & divides the first number by the second \\
        \texttt{\%} & modulus         & remainder after dividing the first number by the second
    \end{tabu}
\end{small}

\par\bigskip


We don't have a \texttt{Math} object in PHP, instead there are just lots of functions - but the naming should be familiar:

\begin{minted}{php}
    // rounding
    floor(12.3030); // 12
    ceil(12.3030); // 13
    round(12.3030); // 12

    // powers/roots
    pow(2, 3); // 8
    sqrt(16); // 4

    // random
    mt_rand(5, 10); // random integer between 5 and 10 (inclusive)

    // trigonometry
    cos(1.2); // 0.362... takes an angle *in radians*
    asin(0.3); // 0.304...returns value *in radians*
    tanh(3); // 0.995... hyperbolic tangent
\end{minted}


But we can still run into weird issues with numbers in PHP. As a general rule \textit{never trust floating point numbers}

\begin{minted}{php}
    floor((0.1 + 0.7) * 10); // 7 - oop!
\end{minted}


\section{Strings}

Strings are also very similar to JavaScript.
\\

You can use single- or double-quotes:

\begin{minted}{php}
    $firstName = "Casper";
    $lastName = 'Spoooky';
\end{minted}

Double-quotes allow you to interpolate values:

\begin{minted}{php}
    $fullName = "{$firstName} {$lastName}";
\end{minted}

We use curly-braces to enter interpolation mode and then use the variable name inside. The \texttt{\$} here is part of the variable, whereas in JavaScript interpolation it's part of the syntax for interpolation (so the dollar is \textit{outside} the curly-braces).
\\

Because interpolation is such a common thing to do, generally we'll use double-quotes for strings.

\subsection{Concatenation}

PHP uses \texttt{.} for concatenation, this avoids the issues that JavaScript had with the overloaded \texttt{+} operator. It can, however, lead to many a brain-fart as you try and use \texttt{+} when you mean \texttt{.}:

\begin{minted}{php}
    var_dump($firstName . $lastName); // "CasperSpooky"
    var_dump($firstName + $lastName); // PHP Warning - A non-numeric value encountered
    var_dump("1" + "2"); // 3 - coerces to numbers
\end{minted}

\subsection{String Functions}

Unlike in JS, strings are not objects in PHP. That means they don't have properties or methods. So we have to use functions to work with them:

\begin{minted}{php}
    strtolower("Blah"); // "blah"
    strtoupper("Blah"); // "BLAH"
    trim("   Blah  "); // "Blah"
    substr("Fishsticks", 4); // "sticks"
\end{minted}

There are many other string function in the \href{http://www.php.net/manual/en/ref.strings.php}{PHP documentation}.


\section{Booleans}

As with JavaScript, PHP has the boolean values \texttt{true} and \texttt{false}. The only difference in PHP is that they're not case sensitive:

\begin{minted}{php}
    $bool = true;
    $bool = True; // also valid
    $bool = TRUE; // still valid
    $bool = TrUe; // totes valid
\end{minted}

For consistency it's best to stick with the lowercase version.

\subsection{Boolean Logic}

PHP has \texttt{\&\&}, \texttt{||}, and \texttt{!} which work in the same way as in JavaScript:

\begin{minted}{php}
    true && false; // false
    true || false; // true
    !true; // false
    !false; // true
\end{minted}

PHP also has the written versions \texttt{and} and \texttt{or}:

\begin{minted}{php}
    true and false; // false
    true or false; // true
\end{minted}

If you're coming from JavaScript, you should use the \texttt{\&\&} and \texttt{||} versions. The other versions have different ``precedence'' and will not always work as you expect.

\subsection{Comparison Operators}

All your favourite comparison operators are back:
\\

\begin{small}
    \begin{tabu}{c l X}
        \textbf{Operator} & \textbf{Name} & \textbf{Description} \\
        \texttt{===} & strict equality & \texttt{true} if the values are the same \\
        \texttt{!==} & non-equality & \texttt{false} if the values are the same\\
        \texttt{<} & less than & \texttt{true} if the first value is less than the second value  \\
        \texttt{>} & greater than & \texttt{true} if the first value is greater than the second value\\
        \texttt{<=} & less than or equal to & \texttt{true} if the first value is less than or equal to the second value  \\
        \texttt{>=} & greater than or equal to & \texttt{true} if the first value is greater than or equal to the second value
    \end{tabu}
\end{small}

\par\bigskip

As with JavaScript there are also the type-coercing \texttt{==} and \texttt{!=} operators, but these are best avoided.


\subsection{Falsy Values}

PHP has all the falsy values that JavaScript has. Empty arrays are also falsy in PHP (which they are \textit{not} in JavaScript):

\begin{itemize}
    \item \texttt{false} itself
    \item The number zero: \texttt{0}, \texttt{0.0}, \texttt{-0}, \texttt{-0.0}
    \item The empty string: \texttt{""}
    \item An empty array: \texttt{[]}
    \item \texttt{NULL}
\end{itemize}



\section{Additional Resources}

\begin{itemize}[leftmargin=*]
    \item \href{https://www.php.net/manual/en/language.operators.arithmetic.php}{PHP: Arithmetic Operators}
    \item \href{https://www.php.net/manual/en/ref.math.php}{PHP: Maths Functions}
    \item \href{http://php.net/manual/en/language.operators.logical.php}{PHP: Logical Operators}
    \item \href{http://php.net/manual/en/language.operators.comparison.php}{PHP: Comparison Operators}
\end{itemize}


\chapter{Control Flow}
\section{Conditionals}

\subsection{\texttt{if} Statements}

\subsection{Ternary Operator}

\subsection{\texttt{switch} Statements}


\section{Loops}

\subsection{\texttt{for} Loops}

\subsection{\texttt{while} Loops}


\section{Additional Resources}

\begin{itemize}[leftmargin=*]
    \item %
\end{itemize}


\chapter{Functions}
Functions in PHP serve the same purpose as functions in JavaScript: they allow us to reuse bits of code.
\\

They are written in the same way as functions were historically written in most C-style language (including JavaScript):

\phpinputminted{01/figures/04/01-function}


\section{Scope}

Because we don't declare variables in PHP it takes a very explicit approach to scope: by default variables used inside functions are assumed to be locally scoped.
\\

The following will cause an error as \texttt{\$message} is assumed to be in local scope:

\phpinputminted{01/figures/04/02-scope-no-global}

This means if we want to access a non-local variable we need to use the \texttt{global} keyword:

\phpinputminted{01/figures/04/03-scope-global}

This can actually be seen as a positive, as it makes it harder to write impure functions by accident.


\section{Anonymous Functions}

Standard functions in PHP are not values like they are in JavaScript, so we can't just pass them round and assign them to variables in quite the same way. However, passing functions around is such a useful thing to be able to do that recent versions of PHP added ``Closures'' as a way to do this:

\begin{minted}{php}
    $add = function ($a, $b) {
        return $a + $b;
    };

    $result = $add(1, 2);
    var_dump($result); // int(3)
\end{minted}

As you can see, because it's stored in a variable we need to use the \texttt{\$} when calling the function.
\\

Some functions in PHP require a \texttt{callable} argument, which just means a function:

\begin{minted}{php}
    $result = array_map(function ($value) {
        return $value * $value;
    }, [1, 2, 3, 4, 5]);

    var_dump($result); // [1, 4, 9, 16, 25]
\end{minted}

Closures don't have access to variables declared outside of themselves. In order to use these you use the \texttt{use} keyword:

\begin{minted}{php}
    $message = "Hello";

    $say = function ($a) use ($message) {
        return "{$message} {$a}";
    };

    $result = $say("Wombat");
    var_dump($result); // string(12) "Hello Wombat"
\end{minted}


\section{Strict Types}

As a general rule functions should only accept arguments of a specific type and return arguments of a specific type: e.g. \texttt{add} should only accept numbers. We can enforce this using ``type declarations''\footnote{In previous versions of PHP these were called ``type hints''}.
\\

First we turn on strict typing:

\begin{minted}{php}
    <?php

    declare(strict_types=1); // always first line after opening tag
\end{minted}

Next we add type declarations to our function:

\begin{minted}{php}
    function add(float $a, float $b) : float {
        return $a + $b;
    }
\end{minted}

Each parameter (\texttt{\$a} and \texttt{\$b}) is given an explicit type. We also add a ``return'' type - the type of value the function returns - after the parameter brackets.
\\

The parameter types and return types needn't match. Here's a function that takes a string (\texttt{\$str}) and integer (\texttt{\$int}) as parameters and returns a string:

\begin{minted}{php}
    function repeat(string $str, int $times) : string {
        $output = "";

        for ($i = 0; $i < $times; $i += 1) {
            $output .= $string;
        }

        return $output;
    }
\end{minted}

The possible type declarations are:

\begin{itemize}
    \item \texttt{int}: for numbers that have to be whole (e.g. a limit of a \texttt{for} loop)
    \item \texttt{float}: for any numbers (an integer passed to \texttt{float} \textit{will} work)
    \item \texttt{string}
    \item \texttt{bool}
    \item \texttt{array}
    \item \texttt{callable}: a closure
\end{itemize}

If you try and call a function and pass in/return the wrong type of value PHP will throw an error.


\section{Additional Resources}

\begin{itemize}[leftmargin=*]
    \item \href{http://www.php.net/manual/en/functions.user-defined.php}{PHP: Functions}
    \item \href{http://www.php.net/manual/en/language.types.callable.php}{PHP: Callable}
    \item \href{http://www.php.net/manual/en/functions.arguments.php#functions.arguments.type-declaration.strict}{PHP: Strict Typing}
\end{itemize}


\chapter{Arrays}
Arrays comes in two forms in PHP: numerically indexed and associative.
\\


\section{Numerically Indexed Arrays}

Numerically indexed arrays are exactly the same as in JavaScript.

\begin{minted}{php}
    $values = [1, 2, 3, 4, 5];
    $first = $values[0]; // first item in array
    $last = $values[4]; // last item in this array
    var_dump($first); // 1
    var_dump($last); // 5
\end{minted}

We can add a value to the end of an array:

\begin{minted}{php}
    $values[] = 6;
\end{minted}

We can also find out how many items are in an array:

\begin{minted}{php}
    count($values); // 5
\end{minted}


\section{Associative Arrays}

Associative arrays are effectively PHP's version of object literals: in PHP ``objects'' are always instances of a \texttt{class}, but the key-value pairing of object literals is still a useful concept.


\section{Iterating Over Arrays}

We can use a \texttt{foreach} loop to iterate over every item in an array.



\section{Additional Resources}

\begin{itemize}[leftmargin=*]
    \item \href{https://www.php.net/manual/en/ref.array.php}{PHP: Array Functions}
\end{itemize}


\chapter{Standard Library}
\section{Dates}

\section{Random Numbers}




\chapter{Regex}
Regular Expressions (or ``regex'' for short) are a way to check/search a string for a \textit{pattern} as opposed to a specific string.
\\

They're a little bit mad looking to start with. In fact they're a little bit mad looking even after you've been using them for years. But they are very useful as long as you don't get too carried away.
\\

For example, we might want to split a string on a comma followed by \textit{any} number of spaces (not just one). We'd write that using the following regex:

\begin{minted}{text}
    /,\s*/
\end{minted}


Or if we wanted to match any combination of lowercase letters, numbers, underscores, and hyphens between 3 and 16 characters long we could write this with the following regex:

\begin{minted}{text}
    /^[a-z0-9_-]{3,16}$/
\end{minted}

JavaScript also supports regexes (it's actually it's own \textit{type} in JS, like numbers and strings).

\pagebreak


\section{Parts}

We're not going to get into every aspect of Regexes, but we'll cover enough for them to be useful.

\subsection{Literal Strings}

The first thing to be aware of is that a RegEx of just non-special characters represents that string. So \texttt{wombat} is just the string ``wombat''.

\subsection{Quantifiers}

Quantifiers allow us to specify that a character should appear zero, one, or many times.

\begin{center}
    \begin{small}
        \begin{tabu}{r l}
            \textbf{Quantifier}   & \textbf{Description} \\
            \texttt{*}          & 0 or more \\
            \texttt{+}          & 1 or more \\
            \texttt{?}          & 0 or 1 \\
            \texttt{\{3\}}      & exactly 3 \\
            \texttt{\{3,\}}     & 3 or more \\
            \texttt{\{3,5\}}   & 3, 4, or 5
        \end{tabu}
    \end{small}
\end{center}

For example:

\begin{minted}{text}
    /a+/        - would match 'a', 'aa', 'aaa', 'aaaa', etc.
    /b*/        - would match '', 'b', 'bb', 'bbb', etc.
    /c?/        - would match '' and 'c'
    /d{3}/      - would match 'ddd'
    /d{3,5}/    - would match 'ddd', 'dddd', and 'ddddd'
    /abc*/      - would match 'ab', 'abc', 'abcc', 'abccc', etc.
    /(abc)*/    - would match '', 'abc', 'abcabc', 'abcabcabc', etc.
    /https?:/   - would match 'http:' and 'https:'
\end{minted}


\subsection{Groups \& Ranges}

Groups and ranges allow us to specify a range of characters that we're interested in.

\begin{center}
    \begin{small}
        \begin{tabu}{r l}
            \textbf{Range}          & \textbf{Description} \\
            \texttt{[a-z]}          & all lowercase letters \\
            \texttt{[A-Z]}          & all uppercase letters \\
            \texttt{[0-9]}          & all numbers \\
            \texttt{[abc]}          & 'a', 'b', or 'c' \\
            \texttt{[0-9afg]}       & all numbers plus 'a', 'f', and 'g' \\
            \texttt{[a-zA-Z]}       & all letters, case-insensitive \\
            \texttt{[a-zA-Z0-9]}    & alphanumeric characters \\
            \texttt{[0-9A-F]}       & valid hexadecimal digits \\
            \texttt{[\textasciicircum abc]}         & \textbf{not} 'a', 'b', or 'c'
        \end{tabu}
    \end{small}
\end{center}

A group is a set of characters wrapped with square brackets. A range, which must always be used \textit{inside} a group, represents a series of characters, e.g. all the letters between \texttt{a} and \texttt{z}.
\\

For example:

\begin{minted}{text}
    /[a-z]+/          - any number of lowercase letters
    /[a-z0-9_-]/      - a single lowercase letter, digit, underscore, or hyphen
\end{minted}

\subsection{Special Characters}

These represent special characters like tab and a new line:

\begin{center}
    \begin{small}
        \begin{tabu}{r l}
            \textbf{Character}          & \textbf{Description} \\
            \texttt{\textbackslash n}          & a new line \\
            \texttt{\textbackslash r}          & a carriage return \\
            \texttt{\textbackslash t}          & a tab \\
        \end{tabu}
    \end{small}
\end{center}

These can generally be used in regular strings too.

\subsection{Character Classes}

Character classes are shortcuts for specific ranges.

\begin{center}
    \begin{small}
        \begin{tabu}{r l}
            \textbf{Class}                    & \textbf{Description} \\
            \texttt{\textbackslash s}         & whitespace \\
            \texttt{\textbackslash S}         & \textbf{not} whitespace \\
            \texttt{\textbackslash w}         & word (\texttt{[A-Za-z0-9\_]}) \\
            \texttt{\textbackslash W}         & \textbf{not} word \\
            \texttt{\textbackslash d}         & digit \\
            \texttt{\textbackslash D}         & \textbf{not} digit
        \end{tabu}
    \end{small}
\end{center}

For example:

\begin{minted}{text}
    /,\s*/          - a comma followed by 0 or more spaces
    /\w\s+\w/       - two words separated by 1 or more spaces
\end{minted}


\subsection{Dot}

In a regex the \texttt{.} character has a special meaning: \textit{any} character except for \texttt{\textbackslash n}. You need to be careful using it, particularly with the \texttt{*} and \texttt{+} quantifiers.
\\

\begin{minted}{text}
    /.+@.+/         - an '@' symbol with some number of other characters either side
\end{minted}

If you want to match an actual full stop you need to ``escape'' it with a backslash:

\begin{minted}{text}
    /\.+@\.+/       - an '@' symbol with some number of '.' either side
\end{minted}



\subsection{Anchors}

Sometimes \textit{where} the substring appears is important.

\begin{center}
    \begin{small}
        \begin{tabu}{r l}
            \textbf{Anchor}             & \textbf{Description} \\
            \texttt{\textasciicircum}   & beginning of the string \\
            \texttt{\$}                 & end of the string \\
        \end{tabu}
    \end{small}
\end{center}

For example:

\begin{minted}{text}
    /^abc/          - would match 'abc' but not '0abc'
    /abc$/          - would match 'abc' but not 'abc0'
\end{minted}


\section{Regex with PHP}

We can use regexes for all sorts of string manipulations. The three most common are:

\begin{itemize}
    \item Searching a string
    \item Splitting a string
    \item Replacing a string
\end{itemize}


\subsection{\texttt{preg\_match}}

The \texttt{preg\_match} function can be used to check if a string matches a regular expression:

\phpinputminted{02/figures/01/01-preg-match}

It returns \texttt{1} if a match is found and \texttt{0} if it is not. Make sure you always use \texttt{===} when checking the result, as it returns \texttt{false} if an error occurs - which might get confused for \texttt{0} if you use \texttt{==}:

\begin{minted}{php}
    if (preg_match("/l+/", "Hello") === 1) {
        // matches one or more 'l' characters
    }
\end{minted}


\subsection{\texttt{preg\_split}}

\texttt{preg\_split} can be used to split a string on a certain regex:

\begin{minted}{php}
    $csv = "first, second,   third,fourth";

    // split on a comma followed by 0 or more spaces
    $result = preg_split("/,\s*/", $csv);

    // [
    //    [0] => "first",
    //    [1] => "second",
    //    [2] => "third",
    //    [3] => "fourth"
    // ]
\end{minted}

We pass it a regex and a string and it gives us back an array of strings where the original string has been split on the regex.

\subsection{\texttt{preg\_replace}}

The \texttt{preg\_replace} function can be used to replace part of a string that matches a regex with something else:

\begin{minted}{php}
    $str = 'blah      blah   blah';

    // replace one or more space with a single space
    $tidied = preg_replace("/\s+/", " ", $str);

    // "blah blah blah"
\end{minted}


There is a lot more to \texttt{preg\_replace} than this basic example,\footnote{``Back references'' are particularly useful} but we'll keep it simple for now.

\begin{infobox}{Flags}
    In PHP we can add various \textbf{flags} to the end of the regex. These go after the last forward-slash, e.g. \texttt{"/[a-z]*/i"}.
    \\

    There are three particularly useful ones in PHP:

    \begin{tabu}{r l X}
        \textbf{Flag}    & \textbf{Name}    & \textbf{Description} \\
        \texttt{i}       & case insensitive & pattern will match upper and lower case \\
        \texttt{m}       & multi-line       & separate lines count as separate strings for anchors\\
        \texttt{s}       & dot all          & the \texttt{.} character should include new lines\\
    \end{tabu}
\end{infobox}


\section{Alternatives to Regex}

For basic validation you are often better using PHP's \texttt{filter\_var} function:

\begin{minted}{php}
    $email = "penny@hello.horse";
    $valid = filter_var($email, FILTER_VALIDATE_EMAIL);

    if ($valid) {
        // valid email address
    }
\end{minted}

The \texttt{filter\_var} function takes a string and a filter type. It then returns the filtered string if it is valid or \texttt{false} otherwise.
\\

Here are some particularly useful filters:\footnote{These big shouty looking things are constant variables defined by PHP - in C-based languages constants are often written in uppercase with underscores}

\begin{itemize}
    \item \texttt{FILTER\_VALIDATE\_EMAIL}
    \item \texttt{FILTER\_VALIDATE\_DOMAIN}
    \item \texttt{FILTER\_VALIDATE\_URL}
\end{itemize}

There's a full list \href{https://www.php.net/manual/en/filter.filters.validate.php}{on the PHP docs}.



\section{The Dangers of Regex}

\quoteinline{Some people, when confronted with a problem, think ``I know, I'll use regular expressions.'' Now they have two problems.}{Jamie Zawinski}

It's not uncommon for people new to programming to try and solve complex string manipulations using regular expressions. This can lead to hard to read and inefficient code. There are many problems that require a \textbf{parser}: a much more clever sort of algorithm that can elegantly cope with things like matching start/end tags.
\\

As a general rule, if your regular expression isn't easy to understand in one glance, then you probably shouldn't be using them.



\section{Additional Resources}

\begin{itemize}[leftmargin=*]
    \item \href{https://regexr.com}{Regexr}: an online Regex testing tool - make sure you set it to use ``PCRE''
    \item \href{http://www.php.net/manual/en/function.preg-match.php}{PHP: \texttt{preg\_match}}
    \item \href{http://www.php.net/manual/en/function.preg-match-all.php}{PHP: \texttt{preg\_matchall}}
    \item \href{http://www.php.net/manual/en/function.preg-replace.php}{PHP: \texttt{preg\_replace}}
    \item \href{http://www.php.net/manual/en/function.preg-split.php}{PHP: \texttt{preg\_split}}
    \item \href{https://www.freecodecamp.org/learn/javascript-algorithms-and-data-structures/regular-expressions/using-the-test-method}{freeCodeCamp: RegEx Exercises with JavaScript}
    \item \href{https://blog.codinghorror.com/regular-expressions-now-you-have-two-problems/}{Regular Expressions: Now You Have Two Problems}
    \item \href{http://www.regexcrossword.com/}{RegEx Crossword}
    \item \href{https://stackoverflow.com/a/1732454}{Stack Overflow: RegEx match open tags except XHTML self-contained tags}
\end{itemize}


\chapter{Classes in PHP}
A class is an abstract representation of an object that you want to create. For example, you might have a class \texttt{Person} that allows you to create lots of object \textbf{instances} representing different people.
\\

Here's a class that represents a person:

\inputminted{php}{02/figures/02/01-class.php}

As you can see, it's much the same as a JavaScript class: we have the \texttt{class} keyword followed by the name of the class and there are some functions inside the class.
\\

As in JavaScript, we call functions that belong to an object \textbf{methods} and the values \textbf{properties}.
\\

However, there are some things we've not seen in JavaScript: type declarations, the words \texttt{private} and \texttt{public}; and we declare our properties outside of the constructor method. We'll look at these in more detail shortly.
\\

Here's how we'd use our class:

\phpinputminted{02/figures/02/02-class-usage}

You can see that where we'd write a dot in JavaScript (\texttt{jim.getAge()}), we write an arrow in PHP (\texttt{\$jim->getAge()}), but otherwise it's almost identical in usage.

\begin{infobox}{PSR-2: Coding Style Guide}
    You've possibly noticed that in all the examples above the opening curly brace (\texttt{\{}) for classes and methods is on its own line. This is part of the \href{https://www.php-fig.org/psr/psr-2/}{PSR-2: Coding Style Guide} spec.
    \\

    If you do an \texttt{if} statement (or other control structure) then the opening curly brace, obviously, goes on \textit{the same} line.
    \\

    You're probably thinking that this doesn't make the slightest bit of sense. And you'd be right. PSR-2 was created by sending round a questionnaire about coding style to 30 or so of the most prolific PHP programmers and they just went with whatever the majority said for each point.
    \\

    But it's the style that everyone uses now. You'll get used to it.
\end{infobox}


\section{\texttt{\$this}}

Inside our classes we can use the \texttt{\$this} keyword to access properties and methods that belong to the current object instance. It works in much the same way as JavaScript except that it's much more reliable: \texttt{\$this} in PHP \textit{always} refers to the current object and has no meaning elsewhere.

\phpinputminted{02/figures/02/04-this}


\subsection{Returning \texttt{\$this}}

If your method doesn't have anything to return, for example if it just sets a value, then you can return \texttt{\$this}: it will give back the current object instance to the user, meaning that they can \textbf{chain} such methods together:

\phpinputminted{02/figures/02/05-returning-this}

For now we'll leave the return type for these methods blank, but when we look at object-oriented programming later we'll see that we can add useful information here.


\begin{infobox}{Using \texttt{new} Objects}
    Unlike in JavaScript you can't immediately use a created object:

    \begin{minted}{php}
        // won't work
        new Person("Jim", "Henson", "1936-09-24")->getAge();
    \end{minted}

    However, if you really want to, you can get around this with a pair of brackets around the \texttt{new} statement:

    \begin{minted}[frame=topline]{php}
        // will work
        // but you don't have a reference to the object anymore
        (new Person("Jim", "Henson", "1936-09-24"))->getAge();
    \end{minted}
\end{infobox}




\section{Properties}

In PHP we declare all of the properties that our class uses at the top of the class. This makes it easy to see which values are available. It also allows us to set default values easily:

\phpinputminted{02/figures/02/03-default-properties}

\begin{infobox}{Typed Properties}
    As of PHP 7.4 (released in late 2019) you can add types to class properties:

    \begin{minted}{php}
      private string $engine = "mysql";
      private string $host = "127.0.0.1";
      private int $port = 3306;
    \end{minted}

    We won't be covering typed properties in any detail, as they're still very new, but you can \href{https://stitcher.io/blog/typed-properties-in-php-74}{read more here}.
\end{infobox}



\section{Visibility}

Methods and properties in PHP can have three levels of visibility: \textbf{public}, \textbf{private}, and \textbf{protected}.
\\

A \textbf{public} method can be called anywhere in the PHP code. A \textbf{public} property can be read and changed from anywhere in the PHP code.
\\

A \textbf{private} method can only be called within the class it is declared in using \texttt{\$this}. A \textbf{private} property can only be read and changed within the class it belongs to by using \texttt{\$this}.

\phpinputminted{02/figures/02/06-visibility}

A \textbf{protected} property/method can only be used within the class it is declared in and any class that inherits from it (we'll look at inheritance later).
\\

In most instances it's good practice to make all of your properties private and then use ``getter'' and ``setter'' methods if you need to be able to change the values outside the class.


\section{Static Methods \& Properties}

Classes are primarily used for creating object instances. But sometimes it's useful to write some functionality about the object type instead of object instances.
\\

For example, if we have a \texttt{Person} class we might want to write a bit of functionality that gives us an alphabetic array of last names. We could write a \texttt{lastNames()} function, but then it's not associated with the \texttt{Person} class.
\\

Instead, we will write a \texttt{static} method: a method that belongs to the class itself rather than to an object instance.

\phpinputminted{02/figures/02/07-static}

Now it is clear that the \texttt{lastNames()} method has something to do with \texttt{Person} objects.
\\

Remember, \texttt{private} properties and methods can only be accessed using \texttt{\$this}, which doesn't have a value in \texttt{static} methods. So you'll need to use getters/setters just as you would if the function was written outside of the class.


\begin{infobox}{Paamayim Nekudotayim}
    The \texttt{::} symbol is also known as the ``Paamayim Nekudotayim'', which is Hebrew for ``double colon''. This can lead to the somewhat mystifying error:

    \begin{minted}{diff}
        PHP expects T_PAAMAYIM_NEKUDOTAYIM
    \end{minted}

    All it's saying is you need a \texttt{::} somewhere.
    \\

    The \href{https://en.wikipedia.org/wiki/Zend_Engine}{Zend Engine}, which was behind PHP 3.0 and all subsequent releases, was originally developed at the Israel Institute of Technology.
\end{infobox}

Sometimes it's useful to be able to refer to the class you're currently working in. We can use the \texttt{static} keyword to do this:

\phpinputminted{02/figures/02/08-static-keyword}

\begin{infobox}{\texttt{static} vs \texttt{self}}
    You will sometimes see \texttt{self} instead of \texttt{static} to reference the current class. Using \texttt{static} in this way was only added in PHP 5.3, so a lot of older code uses \texttt{self}.
    \\

    If you're not using inheritance, then it doesn't make any difference which one you use. If you do then \texttt{self} refers to class that it is written in and \texttt{static} refers to the class it is called in (which might be different from where it was written if you're using inheritance).
\end{infobox}

You can also have \texttt{static} properties. Because they belong to the class they always exist, so you can use them for storing values that you want to have around. This is very useful for caching values.
\\

Say we needed to create a \texttt{\$renderer} object that all our object instances can use to render\textellipsis{} something. We could use a \texttt{static} property to store it, so that we only create it one time:

\phpinputminted{02/figures/02/09-static-property}

Privately declared \texttt{static} variables are accessible from inside object instances.


\section{Additional Resources}

\begin{itemize}[leftmargin=*]
    \item \href{https://phpapprentice.com/classes.html}{PHP Apprentice: Classes}
    \item \href{https://laracasts.com/series/php-for-beginners/episodes/12}{Laracasts: Classes 101}
    \item \href{https://laracasts.com/series/object-oriented-bootcamp-in-php/episodes/1}{Laracasts: Classes}
    \item \href{https://phpapprentice.com/static.html}{PHP Apprentice: Static}
\end{itemize}




\chapter{Namespaces}
So far we've had to put all of our classes into a single file, when ideally we'd like \textit{one class per file}.\footnote{It makes your code much more reusable}

\section{\texttt{require\_once}}

We can import one PHP file into another using the \texttt{require\_once} keyword:

\php{}{03/figures/01/01-require-once}

We give \texttt{require\_once} a relative file path and it will be as if the contents of that file are included in place.

\pagebreak

\begin{infobox}{\texttt{require} and \texttt{include}}
    There is also the \texttt{require} command: this does the same as \texttt{require\_once}, except you could accidentally load the same file in more than once - which you almost never want to do.
    \\

    There is also \texttt{include} and \texttt{include\_once} which do the same as the \texttt{require} equivalents, except if it can't find the file it will keep running and just show an error message. However, normally if we're trying to include a file we wouldn't want the code to run at all if it can't find the file, so \texttt{require} is preferred.
\end{infobox}



\section{Naming Collisions}

Large PHP apps can have hundreds (or even thousands) of classes. It's not uncommon for two classes to end up with the same name. For example, in a blog app you might have a \texttt{Post} class which deals with the data for each post on the site. But you might also have a \texttt{Post} class which posts an update to Slack each time a post is added.
\\

We could, of course, be careful about the naming of each class, calling one \texttt{BlogPost} and the other \texttt{SlackPost}, but in large apps it can be tricky keeping track of every class name that you've used - and it becomes practically impossible when you have multiple developers working on the same app.
\\

Even if we're really careful naming our classes, we don't have any control over the names of classes in PHP libraries that other people have written. It would be unrealistic to make sure that none of the class names you've used clash with those in any libraries that you might use.


\begin{infobox}{The Bad Old Days}
    I lied just then. Back in the before-times, when PHP was still trying to find itself, you \textit{did} have to use unique names for every single class - including the ones in libraries (which you had no control over). To get around this issue you would pick an almost definitely unique prefix (like your company name) and add it to the front of every single class in the app: \texttt{SmallHadronCollider\_BlogPost}, \texttt{SmallHadronCollider\_SlackPost}. Needless to say, this made the code where the classes were used very messy.
\end{infobox}



\section{Namespaces}

\textbf{Namespaces} were added to PHP 5.3 to avoid this problem. The most everyday use of namespaces is the file system on your computer: you can have two files called exactly the same thing \textit{as long as they're in separate directories}.
\\

Namespacing in PHP is much the same idea. We assign each class to a namespace and then we can have two classes with the same name, \textit{as long as they're in separate namespaces}. This means that when we use the class we need to tell PHP which namespace we are talking about.
\\

We assign a namespace by adding a \texttt{namespace} declaration at the top of the file:

\begin{minted}{php}
    namespace Blog\Data;

    class Post { ... }
\end{minted}

Now, when we want to use this class we'll need to use the namespace:

\begin{minted}{php}
    new Blog\Data\Post();
\end{minted}

This might not seem any better than the old way of doing things (i.e. using \texttt{BlogPost}), but PHP also gives us the \texttt{use} keyword.
\\

We can put a \texttt{use} statement at the top of a PHP file to tell it to always use a specific namespaced class:

\begin{minted}{php}
    use Blog\Data\Post;

    // further down the file
    new Post(); // actually new Blog\Data\Post()

    // we can use the other namespaced Post class
    // we just need to use the full namespace
    new Services\Slack\Post();
\end{minted}

Generally we'll use the same class multiple times inside a file, so this saves a lot of typing.
\\

If the class you want to use is in the \textit{same} namespace as the current class you don't even need a \texttt{use} statement.
\\

You can \textbf{alias} a class to give it a different name in the file you're working in. This can be particularly useful if you have two classes which share the same class name but are in different namespaces:

\begin{minted}{php}
    use Blog\Data\Post;
    use Services\Slack\Post as SlackPost;

    new Post(); // actually new Blog\Data\Post()
    new SlackPost(); // actually new Services\Slack\Post()
\end{minted}


\begin{infobox}{The static \texttt{class} property}
    Sometimes it's useful to get the fully namespaced name of a class as a string (e.g. in a configuration file). All classes have a static \texttt{class} property:

    \begin{minted}{php}
        $class = Services\Slack\Post::class;
        var_dump($class); // string(19) "Services\Slack\Post"
    \end{minted}

    This is particularly useful as backslashes need escaping in strings, meaning if you were to write the string out by hand you'd have to write:

    \begin{minted}{php}
        $class = "Services\\Slack\\Post"; // need to escape every backslash
    \end{minted}

    It also means you'll get an error if you try and use a class that doesn't exist.
\end{infobox}





\section{Autoloading}

When I said earlier that PHP apps can contain thousands of classes you might have thought ``Well that's going to be an awful lot of \texttt{require\_once} statements''. And, in fact, historically that's exactly what you'd have: a file called something like \texttt{load.php} which listed thousands of files. Every time you wanted to add a class you'd need to write it and then make sure you added it to the massive list.
\\

Thankfully, things have moved on since then and PHP supports \textbf{autoloading}. This lets us tell PHP where to find a specific class based on its name and namespace. However, writing this code ourselves is unnecessary because we'd be much better using the Composer package manager to do it for us.

\section{Additional Resources}

\begin{itemize}[leftmargin=*]
    \item \href{https://daylerees.com/php-namespaces-explained/}{PHP Namespaces Explained}
    \item \href{http://php.net/manual/en/language.oop5.autoload.php}{Auto-Loading Classes}
\end{itemize}


\chapter{Composer}
Composer is PHP's \textbf{package manager} (like \texttt{npm} is for JavaScript). It lets us easily add code written by other people to our projects.
\\

One of Composer's responsibilities is to set up autoloading of any libraries that it adds, that way you don't have to manually link to all the files that you use in your project. It's also very easy to setup Composer so it will autoload any classes that \textit{you've} created.


\section{Initialising}

First we need to add Composer. Run the following in the project directory that you want to add Composer to:

\begin{minted}{bash}
    # the -n bit stops it asking you a bunch of questions
    composer init -n
\end{minted}

This will add a \texttt{composer.json} file to your project.



\section{PSR-4 Autoloading}

Next we need to tell Composer to load our classes for us. We're going to use the \href{https://www.php-fig.org/psr/psr-4/}{PSR-4 namespace standard}. Basically, this means that we pick a directory to be the ``root'' of our namespace, and everything from that point on is just based on directory names.
\\

First, create a directory called \texttt{app}. Then edit the \texttt{composer.json} so that it looks like this:

\begin{minted}{json}
{
    "autoload": {
       "psr-4": {"App\\": "app/"}
    },
    "require": {}
}
\end{minted}

Here we've told Composer that any namespace starting with the root \texttt{App} should look for files in the \texttt{app} directory. We could use anything as the root namespace or directory name.
\\

Next, run \texttt{composer dump}: this, somewhat confusingly, generates an autoload file for us\footnote{Technically, it gets rid of the existing autoload file and then creates a new one - hence the name} in a directory called \texttt{vendor} (this is where Composer installs any libraries we might want to use).
\\

We'll also need to create a file in the root of our project that looks like this:

\begin{minted}{php}
require_once __DIR__ . '/vendor/autoload.php';

// ... code that uses the classes
\end{minted}

Now, as long as we stick to the following rules, we won't need to require any other files manually:

\begin{enumerate}
    \item One class per file, where the file name is the same as the class name (case sensitive)
    \item Put all of our classes in the \texttt{App} root namespace
    \item If we add directories inside the \texttt{app} directory (for extra organisation), they add an extra level to the namespace
\end{enumerate}

Because namespaces and classes in PHP are usually capitalised and, with PSR-4, the directory and filenames match the namespace/class naming, all the files and directories inside \texttt{app} will also be capitalised.
\\

For example, if we had a class called \texttt{Post} that just sat inside the \texttt{app} directory, it should be in a file called \texttt{Post.php} and have the namespace \texttt{App\textbackslash Post}. If we had a class \texttt{Post} that did something with Slack we could create a directory \texttt{app/Slack} and then put the file \texttt{Post.php} in it with the namespace \texttt{App\textbackslash Slack\textbackslash Post}.



\section{Libraries}

As well as handling autoloading for us, Composer's main purpose is to let us use bits of code other people have written. Let's take a look at some useful libraries.
\\

You can search for libraries on \href{https://packagist.org}{Packagist} or \href{https://libraries.io/search?languages=PHP}{Libraries.io}.


\begin{infobox}{Composer \& Git}
    You don't want to add the \texttt{vendor} directory to your Git repositories, as it can be easily recreated by running \texttt{composer install}. So make sure you add \texttt{vendor/} to your \texttt{.gitignore} file.
\end{infobox}


\subsection{\texttt{symfony/var-dumper}}

We install this with \texttt{composer require symfony/var-dumper}. We then have access to the \texttt{dump()} function, which is a more useful version of \texttt{echo}:

\begin{minted}[startinline=true]{php}
dump([1, 2, 3, 4]);
/*
    array:4 [
      0 => 1
      1 => 2
      2 => 3
      3 => 4
    ]
*/

dump(new Person("Zazu"));
/*
    App\Person {
      -name: "Zazu"
    }
*/
\end{minted}

It also adds the \texttt{dd()} function (for ``debug and die''), which dumps the result and then immediately stops the PHP. This can be useful if you want to check something half-way through a process. Be careful, if you use \texttt{dd()} nothing after it will run.
\\

See the \href{https://symfony.com/doc/current/components/var_dumper.html}{VarDumper documentation} for more information.


\subsection{\texttt{Illuminate\textbackslash Support\textbackslash Collection}}

This is part of Laravel, which we'll be covering later. It basically lets us handle arrays in a way which isn't utterly horrible.
\\

We install it by running \texttt{composer require illuminate/support}. It's got \href{http://laravel.com/docs/master/collections#available-methods}{tonnes of really useful methods}, but we'll just look at four of them here: our old friends \texttt{map()}, \texttt{filter()}, and \texttt{reduce()}, as well as a very useful one called \texttt{pluck()}.
\\

Generally collection methods return a new collection object. You can turn a \texttt{Collection} back into a standard array by calling its \texttt{all()} method.

\subsubsection{\texttt{filter}}

Filter is almost identical to JavaScript: we pass it an anonymous function that takes each item in the array and returns a boolean value. It returns a new \texttt{Collection} containing all the items for which the function returned \texttt{true}:

\phpinputminted{03/figures/02/01-filter}

\subsubsection{\texttt{map}}

Map is also very similar to JavaScript: we pass it an anonymous function that takes each item in the array and transforms the value somehow. It returns a new \texttt{Collection} where each item has been transformed:

\phpinputminted{03/figures/02/02-map}

\subsubsection{\texttt{reduce}}

Again, reduce is very similar to JavaScript: we pass it an anonymous function that takes the accumulated value and each item in the array. The return value is passed in as the accumulator value for the next iteration. It returns the final accumulated value:

\phpinputminted{03/figures/02/03-reduce}

Make sure you pass in an initial value for the accumulator, otherwise it will be \texttt{null}, which might cause problems.

\subsubsection{\texttt{pluck}}

We've not come across \texttt{pluck} before, but it's very useful. It assumes your collection contains either associative arrays or objects all with the same structure. You pass it a key value and it extracts a new \texttt{Collection} contain just that key/property from each item in the collection:

\phpinputminted{03/figures/02/04-pluck}



\subsection{\texttt{nesbot/carbon}}

The Carbon library makes working with dates in PHP much easier. We install it by running \texttt{composer require nesbot/carbon}\footnote{If you've already installed \texttt{illuminate/support} this isn't necessary, as it's a dependency for that package}. Once it's installed we have access to the \texttt{Carbon\textbackslash Carbon} class.

\phpinputminted{03/figures/02/05-Carbon}

There are many more features listed in the \href{https://carbon.nesbot.com/docs/}{Carbon documentation}.


\subsection{\texttt{fzaninotto/Faker}}

\href{https://github.com/fzaninotto/Faker}{Faker} is a library that generates fake data for you. This can be useful for ``seeding'' databases: generating test data to make sure that everything's working. We install it with \texttt{composer require fzaninotto/faker}.

\phpinputminted{03/figures/02/06-Faker}

Faker supports \href{https://github.com/fzaninotto/Faker#formatters}{hundreds of different types of data}: IP addresses, credit card numbers, dates, colours, images, \&c.


\begin{infobox}{The Case of the Missing Students?}
    When building the student preparation app for the course we created lots of fake users to make sure everything was working. Then, when real students started signing up, they didn't appear in the admin interface. It took a bit of sleuthing to work out what was going wrong: the query to fetch the students should have been checking for the student ``type'' field, but in actual fact was checking the student's name for the word ``student''. And it just so happened that every single test student that we'd added to the database had the name ``Test Student <number>''! If we'd been using Faker to generate random names we'd have noticed the issue earlier!
\end{infobox}

\section{Additional Resources}

\begin{itemize}[leftmargin=*]
    \item \href{https://www.php-fig.org/psr/psr-4/meta/}{More about PSR-4}
    \item \href{https://getcomposer.org}{Composer}
\end{itemize}




\chapter{Encapsulation}
So far, all of the PHP code you've written has been ``procedural'': start at the top of a file, run through it, maybe call a few functions as you go, and then finish at the end. This is fine for simple programs or when we're just working inside an existing system (e.g. WordPress), but it doesn't really scale to larger applications.
\\

The problem comes because we need to manage \textbf{state}: keeping track of all the values in our code. For a large app you could easily have thousands of values that need storing. Naming and keeping track of all these variables would become a nightmare if they were all in the same global scope.
\\

\textbf{Object-Oriented Programming}\footnote{``Oriented'' not ``Orientated''.} (OOP) is one way to make this easier. The key idea behind OOP is \textbf{encapsulation}: we keep functions and variables that are related to each other in one place (an object) and then use visibility to limit which bits of code can access and change them.
\\

An object is effectively a black box: objects can send \textbf{messages} to each other (by calling methods), but they need not have any knowledge of the inner workings of other objects.


\pagebreak


\begin{infobox}{The Unusual History of PHP}
    PHP has a long and complicated history. The first version of PHP wasn't even a programming language, it was just simple templating language that allowed you to re-use the same HTML code in multiple files.

    \quoteinline{I don't know how to stop it, there was never any intent to write a programming language \textellipsis{} I have absolutely no idea how to write a programming language, I just kept adding the next logical step on the way.}{Rasmus Lerdorf, Creator of PHP}

    Over the years PHP morphed into a simple programming language and then into a modern object-oriented programming language. However, it wasn't really until 2009, with the release of PHP 5.3, that PHP could truly be considered a fully object-oriented language.
    \\

    Because of this gradual change, older PHP frameworks and systems (such as WordPress) were originally written using non-OO code, which is why they still contain a large amount of procedural code.
    \\

    PHP gets a lot of flack for not being a very good programming language and a few years ago that was perhaps a valid criticism. But in recent years, particularly with the release of PHP 7, it's just not true anymore. It certainly still has some issues, but nothing that a few libraries can't get around.

    \quoteinline{There are only two kinds of languages: the ones people complain about and the ones nobody uses}{Bjarne Stroustrup, Creator of C++}
\end{infobox}

\pagebreak

Say that our app includes some code to send an email. If we were using procedural code we would probably have a function called \texttt{sendMail} that we can pass various values to:

\php{}{03/figures/03/01-mail}

But we might want to be able to customise more than just the to, from, and message parts of the email. Which means we'd either need to have a lot of optional arguments (which becomes unwieldy quickly) or rely on global variables:

\php{}{03/figures/03/02-global-mail}

But this is truly horrible: we have no way of preventing other parts of our code from changing these values and we would start having to use long variable names to avoid ambiguity in bigger apps.
\\

So, we want to store the variables and the functionality together in one place and in such a way that values can't be accidentally changed. This is where objects come in:

\php{}{03/figures/03/03-mail-class}

Now if we need to add additional fields, we can just add a property and setter method.


\section{Object-Oriented Programming}

In OOP objects use other objects to get things done. Rather than use variables and functions, pretty much everything is an object instance and we use properties and methods. The key skill of OOP is getting the right objects to talk to one another.
\\

For example, say that we have various users on our website and we want to send them our regular mailing list email. In procedural code we'd probably write something like:

\begin{minted}{php}
    $userEmails = getUserEmailsFromDatabase();

    foreach ($userEmails as $address) {
        sendEmail($address, "Subject", "Message");
    }
\end{minted}

In OOP it might look something like this:

\begin{minted}{php}
    $users = Users::all(); // get all the users
    $email = new Email("Subject", "Message"); // create a new email
    $mailing = new MailingList($users); // create a new mailing list
    $mailing->sendEmail($email);
\end{minted}

Each object represents a specific, \textit{encapsulated}, bit of functionality, dealing with just the things that it needs to and nothing else. We still need to glue the objects together, but this code itself is generally inside other objects.



\section{(Almost) Pure OO}

Many object oriented languages \textit{only} use objects. For example in Java everything lives inside a class and you specify which class your app should create first when you run it.
\\

Because of PHP's history we always need a little bit of procedural code to get our objects up and running. This is often called the \textbf{bootstrap} file.

\php{bootstrap.php}{03/figures/03/04-bootstrap}

Once we've created our first object the idea of object-oriented programming is that we use objects from that point onwards.


\section{The Law of Demeter}

The ``Law of Demeter'' is a guideline for OOP about how objects should use other objects. Expressed succinctly:

\begin{center}
    \textit{Each object should only talk to its friends; don't talk to strangers}
\end{center}

In practice, this means that an object should only call methods on either itself or objects that it has been given. You should avoid calling a method which returns an object and then calling a method on that object: it requires too much knowledge about other objects.

\php{}{03/figures/03/07-Demeter}


\section{Types}

We've already looked at ``scalar types'' in PHP: integers, floats, strings, booleans, arrays, \&c.
\\

In OOP we generally talk of objects as having the type of their class: e.g. a \texttt{Person} object instance is of the \textit{type} \texttt{Person}.
\\

PHP supports \textbf{type declarations} for object instances too. These let us say that the values passed to and returned from a function or method must be instances of a certain type.
\\

For example, say we had a \texttt{MailingList} class with a \texttt{sendWith()} method. It would be useful to say that we can only pass \texttt{Mail} objects into this method\footnote{This is an example of \textbf{dependency injection}.}:

\php{}{03/figures/04/01-MailingList}

Before accepting the \texttt{\$mailer} parameter, we add the type declaration/hint of \texttt{Mail}. Now, if the user of that class tries to pass in something that isn't an instance of \texttt{Mail}, PHP will throw an error.

\php{}{03/figures/04/02-type-error}


\section{Why?}

Adding type declarations like this lets us find errors in our code quickly.
\\

For example, say that we didn't add the \texttt{Mail} type declaration and then passed in an object that didn't have a \texttt{send} method. We'd get an error from PHP, but it would say the error was inside \texttt{MailingList} when we use the \texttt{send} method.

\begin{minted}{text}
    Call to undefined method Person::send()
\end{minted}

But that's not actually the issue, the problem is that we've passed in the \textit{wrong} object type - one without a \texttt{send} method. So the error is actually when we pass the wrong type of thing to the \texttt{sendWith} method. By adding type declarations to the \texttt{sendWith} method we can make sure PHP finds the error at the correct point in our code.

\begin{minted}{text}
    TypeError: Argument 1 passed to MailingList::sendWith() must be an instance of Mail
\end{minted}


\section{Additional Resources}

\begin{itemize}[leftmargin=*]
    \item \href{https://en.wikipedia.org/wiki/Encapsulation_(computer_programming)}{Wikipedia: Encapsulation}
    \item \href{https://stackify.com/oop-concept-for-beginners-what-is-encapsulation/}{What is Encapsulation?}: Uses Java for examples, but largely applicable in PHP
    \item \href{https://en.wikipedia.org/wiki/Law\_of\_Demeter}{Wikipedia: The Law of Demeter}
\end{itemize}


\chapter{Polymorphism}
You might look at the \texttt{MailingList} example from the previous chapter and think, ``What's the point of passing in the \texttt{Mail} object?'' And you'd be right. If we can only pass in a \texttt{Mail} object, then we may as well just create it in the \texttt{MailingList} class.
\\

But what if we wanted to be able to sometimes send our mailing list emails using the server's built-in mail program and sometimes using MailChimp? We can't have a type declaration for two different classes, but if we don't limit the type at all then you could accidentally pass in something that will break your code completely.
\\

This is where the idea of \textbf{polymorphism} comes in. Polymorphism is when two \textit{different} types of object share enough in common that they can take each other's place in a specific context.
\\

There are two ways to enforce this in most OO languages:

\begin{itemize}
    \item \textbf{Interfaces}: when an object implements a defined set of methods.
    \item \textbf{Inheritance}: when an object can inherit methods/properties from another object, creating a hierarchy of object types.
\end{itemize}



\section{Interfaces}

One way we can take advantage of polymorphism is to use ``interfaces''. An \textbf{interface} is a list of \textbf{method signatures} that a class can say it conforms to. It is a contract: if a class implements an interface then we are guaranteed that it has a certain set of methods taking a specified set of arguments.
\\

For example, rather than creating a \texttt{Mailer} abstract class, we could instead create an interface:

\php{MailerInterface.php}{04/figures/01/15-MailerInterface}

You can see that we list out all of the methods that a class that implements this interface \textit{must} implement. We would use it as follows:

\php{Mail.php}{04/figures/01/16-Mail-interface}

We would use it in \texttt{MailingList} in exactly the same way:

\php{MailingList.php}{04/figures/01/17-MailingList-interface}

Interfaces are also a type declaration. So, now we could only pass in classes that implement the \texttt{MailerInterface} interface.
\\

A class can \texttt{implement} as many interfaces as it likes:

\php{MailChimp.php}{04/figures/01/18-MailChimp}


\subsection{Message Passing}

If you look at the \texttt{sendWith()} method you'll see that it only uses the \texttt{to()}, \texttt{from()}, and \texttt{send()} methods of the object that gets passed in. Therefore it's \textit{guaranteed} that if the object passed in conforms to the \texttt{MailerInterface}, which defines all three of those methods, then it will work. That's not to say the implementation will necessarily work, but the \texttt{sendWith()} method has access to all the methods that it requires.
\\

This is a core idea in OOP. But one that is often not talked about with beginner level books on OOP. Although it's called \textit{Object}-Oriented Programming, it's not actually the objects that should get the focus, it's the \textbf{messages} that they can send to one another: the methods and their parameters.
\\

The reason interfaces are such a powerful idea is because they focus solely on the messages and don't tell you anything about the implementation. This is important because of encapsulation. If we have to worry about how a specific object does something, then we can't treat it as a black box.
\\

This is also why we try to only use \texttt{private} properties: if a property is \texttt{public} then we need to know about how the internals of the class work.



\section{Inheritance}

Inheritance lets us create a hierarchy of object types, where the \textbf{children} types inherit all of the methods and properties of the \textbf{parent} classes. This allows reuse of methods and properties but allowing for different behaviours.
\\

For example, say that we want to create a \texttt{Mail} and a \texttt{MailChimp} class. Both of these will be responsible for sending an email, so they will have some things in common, but their inner workings will be different. We could create a parent \texttt{Mailer} class:

\php{Mailer.php}{04/figures/01/07-Mailer}

We move all of the shared code into the \texttt{Mailer} class. We don't put the \texttt{send()} method into the \texttt{Mailer} class, as it will be different for each implementation. We can then \textbf{extend} the \texttt{Mailer} class to copy its behaviour into \texttt{Mail} and \texttt{MailChimp}:

\php{Mail.php}{04/figures/01/08-Mailer-children}

Now, in the \texttt{MailingList} class we can use \texttt{Mailer} as the type declaration. That means that depending on our mood,\footnote{Or possibly something more concrete} we can send messages with either the local mail server or with MailChimp:

\php{MailingList.php}{04/figures/01/09-MailingListRedux}


\subsection{Abstract Classes}

But you can perhaps see a few issues with this. Firstly, you could create an instance of \texttt{Mailer} and pass that into \texttt{sendWith()}. This would cause an issue because the \texttt{Mailer} class doesn't have a \texttt{send()} method, so you'd get an error. Secondly, there's no guarantee that a child of \texttt{Mailer} has a \texttt{send()} method: we could easily create a child of \texttt{Mailer} but forget to add it. This would lead to the same issue:

\php{}{04/figures/01/10-Mailer-oops}

This is where \textbf{abstract classes} come in. These are classes that are not meant to be used to create object instances directly, but instead are designed to be the parent of other classes. We can setup properties and methods in them, but they can't actually be constructed, only extended.
\\

We can also create \texttt{abstract} method signatures. These allow us to say that a child class \textit{has} to implement a method with the given name and parameters. We'll get an error if we don't.
\\

This gets round both issues: we won't be able to instance \texttt{Mailer} and we can make sure any of its children have a \texttt{send()} method:

\php{Mailer.php}{04/figures/01/11-Mailer-redux}


\subsection{Overriding}

Children classes can \textbf{override} methods and properties from parent classes. That means if we wanted to make it so that the \texttt{to()} method in the \texttt{MailChimp} class did something slightly different, then we could write a different \texttt{to()} method. As long as it has \textit{the same method signature}\footnote{The \texttt{\_\_construct()} method is an exception to this rule. As it is unique to the specific class it can have a different set of parameters.} (i.e. excepts the same parameters), this will work:

\php{MailChimp.php}{04/figures/01/12-override}

If necessary it's possible to stop a child class from overriding a method by adding the \texttt{final} keyword in front of it:

\php{Mailer.php}{04/figures/01/13-Mailer-final}

\subsection{\texttt{parent}}

Sometimes when we're overriding a method it can be useful to still have access to the parent object's version. For example, we might want to override the \texttt{to()} method of \texttt{Mailer}, but only to add a bit of functionality. We can do this by calling the method name on \texttt{parent}. Make sure you pass along the arguments when you do this:

\php{MailChimp.php}{04/figures/01/14-parent}

You can call the parent's constructor method with \texttt{parent::\_\_construct()}.



\section{Inheritance Tax}

\quoteinline{The object-oriented version of ``Spaghetti code'' is, of course, ``Lasagna code'': too many layers}{Roberto Waltman}

If you read any books about OOP they'll focus a lot of their time on inheritance. While inheritance can be very useful, all of this attention means that it's often the technique that programmers reach for when they need to add the same bit of functionality to multiple classes. And it will almost always lead to much more complicated code.
\\

That's not to say it isn't useful. It's totally fine to inherit code that frameworks or libraries provide, as in these cases you're generally adding one tiny bit of functionality to something that's much more complicated under-the-hood.\footnote{Although some purists would say even in this case there are better alternatives: see the \href{https://www.thoughtfulcode.com/orm-active-record-vs-data-mapper/}{Active Record vs Data Mapper} debate} But if it's code that you've written, it's always worthwhile thinking ``Do I really need to use inheritance?''
\\

Sandi Metz\footnote{She's been doing OOP since it was invented in the 70s, so she probably knows what she's talking about} suggests not using inheritance until you have \textit{at least} three classes that are \textit{definitely} all using exactly the same methods. You should never start out by writing an abstract class: write the actual use-cases first and only write an abstract class if you definitely need one. If you do use inheritance then try not to create layers and layers of it: maybe have a rule that you'll only ever inherit through one layer.



\section{Composition}

\quoteinline{Prefer composition over inheritance}{The Gang of Four, \textit{Design Patterns}}

The idea of \textit{composition over inheritance} is that rather than sharing behaviour with inheritance, we share it using interfaces and shared classes. If a lot of classes share the same implementation of a method, rather than using inheritance consider moving it into a separate class that they can all share. It might require a bit more code to get working, but it is much easier to make changes to.


\pagebreak

\section{Additional Resources}

\begin{itemize}[leftmargin=*]
    \item \href{https://phpapprentice.com/interfaces.html}{PHP Apprentice: Interfaces}
    \item \href{https://phpapprentice.com/classes-inheritance.html}{PHP Apprentice: Inheritance}
    \item \href{https://en.wikipedia.org/wiki/Composition\_over\_inheritance}{Wikipedia: Composition over Inheritance}
    \item \href{https://www.poodr.com}{Practical Object-Oriented Design in Ruby}: An intermediate book about OOP. In Ruby, but all the key concepts work in PHP too.
    \item \href{https://en.wikipedia.org/wiki/SOLID}{Wikipedia: The SOLID Principles of OOP}: Quite advanced, but incredibly useful if you can get your head round it.
    \item \href{https://www.youtube.com/channel/UCk3yOoaVtORwXipuLZ3jWNg}{YouTube: Sandi Metz Videos}: If you're doing OOP in a few years time, watch all of these. There's a lifetime of experience in these talks.
\end{itemize}




\nchapter{Glossary}
\begin{itemize}[leftmargin=*]
    \item
        \textbf{Class}:
        An abstract representation of an object instance
    \item
        \textbf{Composer}
        PHP's package management system
    \item
        \textbf{Dependency Injection}
        Rather than hard-coding a dependency on a specific class, taking advantage of polymorphism and passing in a class that implements a specific interface.
    \item
        \textbf{Encapsulation}
        Keeping the inner-workings of a class private so that they can only be accessed by sending appropriate messages (by calling methods)
    \item
        \textbf{Inheritance}
        A way to share code between different classes. Should be used with caution as it breaks aspects of encapsulation.
    \item
        \textbf{Instance}:
        An object is an instance of a specific class with its own set of properties
    \item
        \textbf{Interface}:
        A list of method signatures that tell us how to talk to an object that implements it
    \item
        \textbf{Namespace}:
        A set of classes where each class has a unique name. We can have two classes with the same name as long as they are in different namespaces.
    \item
        \textbf{Polymorphism}:
        When two different classes can be used interchangeably in a specific context because they share the relevant method signatures
    \item
        \textbf{Single Responsibility Principle}
        The principle that an object/class should only do one sort of thing. We get more complex behaviour by composing different objects – similar to function composition in functional programming languages.
    \item
        \textbf{Static}:
        A property or method that belongs to a class rather than an object instance
\end{itemize}



\input{../../templates/colophon.tex}

\end{document}
