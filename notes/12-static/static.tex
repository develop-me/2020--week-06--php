Classes are primarily used for creating object instances. But sometimes it's useful to write some functionality about the object type instead of object instances.
\\

For example, if we have a \texttt{Person} class we might want to write a bit of functionality that given an array of \texttt{Person} objects gives us back an array of just last names. We could write a \texttt{lastNames()} function in global scope, but then it's not associated with the \texttt{Person} class.
\\

Instead, we will write a \texttt{static} method: a method that belongs to the class itself rather than to an object instance.

\php{}{12-static/figures/01-static}

Now it is clear that the \texttt{lastNames()} method has something to do with \texttt{Person} objects.


\begin{infobox}{Paamayim Nekudotayim}
    The \texttt{::} symbol is also known as the ``Paamayim Nekudotayim'', which is Hebrew for ``double colon''. This can lead to the somewhat mystifying error:

    \begin{minted}{diff}
        PHP expects T_PAAMAYIM_NEKUDOTAYIM
    \end{minted}

    All it's saying is you need a \texttt{::} somewhere.
    \\

    The \href{https://en.wikipedia.org/wiki/Zend_Engine}{Zend Engine}, which was behind PHP 3.0 and all subsequent releases, was originally developed at the Israel Institute of Technology.
\end{infobox}

\section{The \texttt{static} Keyword}

Sometimes it's useful to be able to refer to the class you're currently working in. We can use the \texttt{static} keyword to do this.
\\

For example, we might want a separate \texttt{static} method that returns the last names alphabetically – but it would be repetitive to rewrite the method we've already got:

\php{}{12-static/figures/02-keyword}

\begin{infobox}{\texttt{static} vs \texttt{self}}
    You will sometimes see \texttt{self} instead of \texttt{static} to reference the current class. Using \texttt{static} in this way was only added in PHP 5.3, so a lot of older code uses \texttt{self}.
    \\

    If you're not using inheritance, then it doesn't make any difference which one you use. If you do, then \texttt{self} refers to class that it is \textit{written} in and \texttt{static} refers to the class it is \textit{called} in (which might be different from where it was written if you're using inheritance).
\end{infobox}


\section{Static Properties}

You can also have \texttt{static} properties. Because they belong to the class they exist for as long as your code is running, so you can use them for storing values that you want to keep around – like global variables, but with a dedicated home. This is very useful for caching values.
\\

Say we needed to create a \texttt{\$renderer} object that all our object instances can use to render\textellipsis{} something. We could use a \texttt{static} property to store it, so that we only create it one time:

\php{}{12-static/figures/03-properties}

\texttt{static} properties can be very useful for things like caching, although there are often other alternatives.


\section{Additional Resources}

\begin{itemize}[leftmargin=*]
    \item \href{https://phpapprentice.com/static.html}{PHP Apprentice: Static}
\end{itemize}
