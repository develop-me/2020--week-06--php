Once our apps get past a certain size it becomes impractical to try and remember every single bit of code you've written. You'll remember that when we were working with functional programming in JavaScript we got around this by \textbf{composing} small bits of functionality to create more complex behaviours and we've already seen how we can use objects to achieve something similar. However, we need to be more careful when using objects, as they are more complicated than individual functions. This is where the concept of \textbf{encapsulation} comes in.
\\

\textbf{Encapsulation} is one of the key ideas behind OOP: we keep functions and variables that are related to each other in an object and then use \textbf{visibility} to limit which bits of code can access and change them.
\\

In this way an object becomes a black box: objects can send \textbf{messages} to each other (by calling methods), but they shouldn't have any knowledge of the inner workings of other objects.


\section{Visibility}

Methods and properties in PHP can have three levels of visibility: \textbf{public}, \textbf{private}, and \textbf{protected}.

\subsection{\texttt{public}}

A \textbf{public} method can be called anywhere in the PHP code.
\\

A \textbf{public} property can be read and changed from anywhere in the PHP code.

\begin{minted}{php}
    class Open
    {
        public $name = "Blah";

        public function doThing()
        {
            // ...does thing
        }
    }

    // elsewhere
    $open = new Open();
    $open->doThing(); // does thing

    dump($open->name); // "Blah"
    $open->name = "Pickle Rick";
    dump($open->name); // "Pickle Rick"
\end{minted}

\subsection{\texttt{private}}

A \textbf{private} method can only be called within the class it is declared in using \texttt{\$this}.
\\

A \textbf{private} property can only be read and changed within the class it belongs to by using \texttt{\$this}.

\begin{minted}{php}
    class Closed
    {
        private $name = "Blah";

        private function doThing()
        {
            // ...does thing
        }

        public function doOtherThing()
        {
            // can access private property with $this
            $name = $this->name;

            // can call private method with $this
            $thing = $this->doThing();

            // ...do other thing
        }
    }

    // elsewhere
    $closed = new Closed();

    // can't call doThing without $this – i.e. outside class
    $closed->doThing(); // PHP Fatal error: Call to private method Closed::doThing()
    $closed->doOtherThing(); // does other thing

    dump($closed->name); // PHP Fatal error: Cannot access private property Closed::$name
    $closed->name = "Pickle Rick"; // PHP Fatal error: Cannot access private property Closed::$name
\end{minted}

\subsection{\texttt{protected}}

A \textbf{protected} property/method can only be used within the class it is declared in and any class that inherits from it. We'll look at inheritance later.


\section{Encapsulation}

We can think of \texttt{public} properties and methods as the surface-area of a class and the \texttt{private} properties and methods as the insides. The smaller the surface-area, the less we have to remember to use the class. Remember, if all our insides are outside then shit goes everywhere.
\\

As a general rule we want to make as much \texttt{private} as possible. Many of your methods will be \texttt{public} (definitely the getters and setters), but if the method is only useful internally then you should make it \texttt{private}.

\pagebreak

In particular there is generally no reason to make properties \texttt{public}: we should write getter and setter methods to access these values:

\begin{minted}{php}
    class Closed
    {
        private $name = "Blah";

        public function setName($name)
        {
            $this->name = $name;
            return $this;
        }

        public function getName()
        {
            return $this->name;
        }
    }

    // elsewhere
    $closed = new Closed();
    dump($closed->getName()); // "Blah"
    $closed->setName("Pickle Rick");
    dump($closed->getName()); // "Pickle Rick"
\end{minted}

This probably seems like a lot of extra code, but it will help us write ``safe'' code.


\section{Writing Safe Code}

Another aspect of encapsulation is making sure that when other bits of code use our classes we do everything we can to stop things going wrong \textit{inside} our class. If the person using our class has to look inside the class to see what's gone wrong, then we've broken encapsulation.
\\

For example, going back to our \texttt{Mail} class, if the \texttt{\$to} property was \texttt{public} it would be easy to set it to \textit{any} value:

\begin{minted}{php}
    $mail = new Mail();
    $mail->to = 12;
\end{minted}

This would then cause issues when we call the \texttt{mail()} method later on: it will try to send an email to a number and probably throw an error. It will look like something has gone wrong \textit{inside} \texttt{Mail}, when really it was the usage of \texttt{Mail} – setting the \texttt{\$to} property to a number – that caused the issue.
\\

If all the properties are \texttt{private} we can protect against this:

\php{}{10-encapsulation/figures/01-Mail}

Now the user of the class \textit{has} to use the \texttt{to()} method – as the property is \texttt{private} – and if they pass in something that is not a valid email address they will get en error message \textbf{that points to where they called \texttt{to()}}. In other words, the error will not be \textit{inside} the \texttt{Mail} class and encapsulation is preserved.

\pagebreak

\begin{infobox}{Exceptions}
    In the above example we \textbf{throw} an \textbf{Exception}. An Exception is just an error that will stop the code from running, just as if we'd written some invalid syntax or called a method that doesn't exist.
    \\

    PHP has \href{https://www.php.net/manual/en/spl.exceptions.php}{various types of custom exception} that you can throw yourself.
    \\

    If you're expecting an error you can also \textbf{catch} exceptions: this allows the code to continue running when things go wrong, but in a controlled fashion.
\end{infobox}


\section{Type Safety}

PHP can do some of the work for us when it comes to checking that the right sorts of values are being passed to our methods.
\\

As a general rule methods should only accept arguments of a specific type and return arguments of a specific type. We can enforce this using ``type declarations''\footnote{In previous versions of PHP these were called ``type hints''}.
\\

For example, we can add type declarations to the \texttt{to()} method:

\begin{minted}{php}
    public function to(string $address) : Mail
    {
        // ...code as above
    }
\end{minted}

The parameter (\texttt{\$address}) is given an explicit type (\texttt{string}). We also add a ``return'' type – the type of value the method returns – after the parameter brackets. In this case the \texttt{to()} method returns \texttt{\$this}, which is an instance of a \texttt{Mail} object, so we use the class name as the return type.
\\

The possible type declarations are:

\begin{itemize}
    \item \textbf{A class name}: an instance of that class
    \item \texttt{int}: for numbers that have to be whole (e.g. a limit of a \texttt{for} loop)
    \item \texttt{float}: for any numbers (an integer passed to \texttt{float} \textit{will} work)
    \item \texttt{string}
    \item \texttt{bool}
    \item \texttt{array}
    \item \texttt{callable}: a closure
\end{itemize}

Now if you try and call a method and pass in/return the wrong type of value PHP will throw an error.

\subsection{\texttt{void}}

If a method doesn't return anything, we can use the special \texttt{void} return type:

\begin{minted}{php}
    // method doesn't return anything
    // so use void return type
    public function mail() : void {
        // ...mail code
    }
\end{minted}

But generally it's better to return \texttt{\$this} if there's no obvious choice.

\begin{infobox}{Strict Typing}
    By default PHP will coerce scalar values to the appropriate type if it sees a type declaration. For example passing \texttt{12} (a number) to the \texttt{to()} method would convert the number into a string (\texttt{"12"}). We can enable ``strict'' typing to disable this behaviour. To do this we add a declaration to the file that \textit{calls} the relevant method:

    \begin{minted}{php}
        <?php

        declare(strict_types=1);
    \end{minted}

    This should go as the very first line after the PHP opening tag (before any \texttt{namespace} declarations).
    \\

    Once strict typing is turned PHP will throw an error if the types do not match.
\end{infobox}


\section{Type Systems}

Being able to create our own ``types'' using classes lets us create \textbf{type systems}: we can start to make guarantees that our code will always work.
\\

Consider the \texttt{send()} method of the \texttt{MailingList} class:

\begin{minted}{php}
  public function send($mail)
  {
    // go through each subscriber one at a time
    foreach ($this->subscribers as $subscriber) {
      // use the passed in Mail object
      // update just the to field each time
      // then send the mail
      $mail->to($subscriber)->mail();
    }
  }
\end{minted}

It currently goes over each subscriber and then calls the \texttt{to()} and \texttt{mail()} methods on the \texttt{\$mail} object that has been passed in. As is, we could pass in \textit{any} type of object to \texttt{send()}:

\begin{minted}{php}
$person = new Person("Mark", "Wales", "1984-04-16");

// oops – that's not going to work
$mailinglist->send($person);
\end{minted}

If we ran the above code we'd get an error inside the \texttt{MailingList} class when we try to call \texttt{to()} on it:

\begin{minted}{text}
    Call to undefined method Person::to()
\end{minted}

However, with one simple change we can restore encapsulation:

\begin{minted}{php}
  // add the Mail type declaration
  public function send(Mail $mail)
  {
    foreach ($this->subscribers as $subscriber) {
      // has to be a Mail object
      // so this is guaranteed to work
      $mail->to($subscriber)->mail();
    }
  }
\end{minted}

Now we've guaranteed that the thing being passed in is of the type \texttt{Mail}. And we know that \texttt{Mail} objects have a \texttt{to()} method that takes a \texttt{string} and returns itself. And we know that a \texttt{Mail} object has a \texttt{mail()} method that doesn't take any arguments. Now if the user tries to pass in the wrong sort of thing:

\begin{minted}{text}
    TypeError: Argument 1 passed to MailingList::send() must be an instance of Mail
\end{minted}

Encapsulation has been restored. Everything is good again\footnote{Everything is \textit{not} good again}.



\section{Additional Resources}

\begin{itemize}[leftmargin=*]
    \item \href{http://www.php.net/manual/en/functions.arguments.php#functions.arguments.type-declaration.strict}{PHP: Strict Typing}
    \item \href{https://en.wikipedia.org/wiki/Information_hiding}{Wikipedia: Information Hiding}
\end{itemize}
