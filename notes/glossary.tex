\begin{itemize}[leftmargin=*]
    \item
        \textbf{Class}:
        An abstract representation of an object instance
    \item
        \textbf{Composer}
        PHP's package management system
    \item
        \textbf{Dependency Injection}
        Rather than hard-coding a dependency on a specific class, taking advantage of polymorphism and passing in a class that implements a specific interface.
    \item
        \textbf{Encapsulation}
        Keeping the inner-workings of a class private so that they can only be accessed by sending appropriate messages (by calling methods)
    \item
        \textbf{Inheritance}
        A way to share code between different classes. Should be used with caution as it breaks aspects of encapsulation.
    \item
        \textbf{Instance}:
        An object is an instance of a specific class with its own set of properties
    \item
        \textbf{Interface}:
        A list of method signatures that tell us how to talk to an object that implements it
    \item
        \textbf{Namespace}:
        A set of classes where each class has a unique name. We can have two classes with the same name as long as they are in different namespaces.
    \item
        \textbf{Polymorphism}:
        When two different classes can be used interchangeably in a specific context because they share the relevant method signatures
    \item
        \textbf{Single Responsibility Principle}
        The principle that an object/class should only do one sort of thing. We get more complex behaviour by composing different objects – similar to function composition in functional programming languages.
    \item
        \textbf{Static}:
        A property or method that belongs to a class rather than an object instance
\end{itemize}
